% PErhatikan urutan abjad kode nomenclature, misal abg, abh, cbe, cbi, menyatakan urutan penulisan di hasil akhir file .pdf. 
%Lihat contoh untuk Singkatan, meskipun urutan penulisan tidak urut abjad, misal ebh mendahului ebe, tetapi hasil akhirnya tetap urut abjad.

% LAMBANG ROMAWI
\nomenclature[abg]{$G$}{laju aliran massa  \nomunit{kg/s}}
\nomenclature[abh]{$h$}{entalpi  \nomunit{J/kg}}
\nomenclature[abq]{\textbf{q}}{fluks kalor  \nomunit{W/m$^2$}}


% LAMBANG YUNANI
\nomenclature[bbe]{$\epsilon$}{emisivitas  \nomunit{--}}
\nomenclature[bbs]{$\Sigma_f$}{tampang lintang fisi makroskopik  \nomunit{1/cm}}
\nomenclature[bbs]{$\Sigma_s$}{tampang lintang fisi hamburan  \nomunit{1/cm}}

% SUBSKRIP
\nomenclature[cbe]{ext}{external}
\nomenclature[cbi]{in}{inlet}
\nomenclature[cbo]{out}{outlet}

% SUPERSKRIP
\nomenclature[dbf]{F}{fuel}
\nomenclature[dbj]{j}{indeks koordinat}


% SINGKATAN
\nomenclature[ebh]{HTGR}{High Temperature Gas-cooled Reactor}
\nomenclature[ebe]{ECCS}{Emergency Core Cooling System}
\nomenclature[ebj]{DTNTF}{Departemen Teknik Nuklir dan Teknik Fisika}
\nomenclature[ebr]{RANS}{Reynold-Average Navier-Stokes}