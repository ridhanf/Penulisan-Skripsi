% PErhatikan urutan abjad kode nomenclature, misal abg, abh, cbe, cbi, menyatakan urutan penulisan di hasil akhir file .pdf. 
%Lihat contoh untuk Singkatan, meskipun urutan penulisan tidak urut abjad, misal ebh mendahului ebe, tetapi hasil akhirnya tetap urut abjad.

% LAMBANG ROMAWI
\nomenclature[aaa]{T${_{db}}$}{Suhu Ruang (\textit{Dry-Bulb Temperature})  \nomunit{$^\circ$C}}
\nomenclature[aab]{RH}{Kelembapan Relatif  \nomunit{\%}}
\nomenclature[aac]{T$_o$}{Suhu Lingkungan (\textit{Dry-Bulb Temeperature}) \nomunit{$^\circ$C}}
\nomenclature[aad]{RD}{Intensitas Radiasi Matahari  \nomunit{W/m$^2$}}
\nomenclature[aae]{AC}{\textit{Setting} AC \nomunit{$^\circ$C}}
\nomenclature[aaf]{HT}{Banyak \textit{heater} menyala \nomunit{ON}}
\nomenclature[aag]{$t$}{Waktu \nomunit{detik}}
\nomenclature[aah]{$f$}{Frekuensi \nomunit{Hertz}}
\nomenclature[aai]{R}{Koefisien Korelasi \nomunit{\%}}
\nomenclature[aaj]{$\mathbb{R}$}{Domain Bilangan Riil}
\nomenclature[aak]{$R(s)$}{Masukan Sistem Kontrol}
\nomenclature[aal]{$C(s)$}{Keluaran Sistem Kontrol}
\nomenclature[aam]{$E(s)$}{Galat Sistem Kontrol}
\nomenclature[aan]{$K$}{Konstanta Pengali}
\nomenclature[aao]{$T(s)$}{Fungsi Pengali Kalang Tertutup}
\nomenclature[aap]{$G(s)$}{Fungsi Pengali Kalang Tertutup Umpan Balik Satuan}
\nomenclature[aar]{$x$}{Lapisan Masukan Jaringan Saraf Tiruan}
\nomenclature[aas]{$y$}{Lapisan Keluaran Jaringan Saraf Tiruan}
\nomenclature[aat]{$z$}{Lapisan Tersembunyi Jaringan Saraf Tiruan}
\nomenclature[abz]{}{}
\nomenclature[aaz]{}{}

% LAMBANG YUNANI
\nomenclature[bba]{$\nu$}{Bobot Jaringan Saraf Tiruan}
\nomenclature[bbt]{$\sigma$}{Fungsi Aktivasi Neuron}

% SUBSKRIP
%\nomenclature[cba]{i}{Iterasi Lapisan Tersembunyi}
%\nomenclature[cbb]{j}{Iterasi Input}
%\nomenclature[cbc]{l}{Iterasi Neuron}
\nomenclature[cbd]{\textit{steady-state}}{Kondisi-Ajeg}

% SUPERSKRIP
\nomenclature[dbj]{n}{Dimensi n}
\nomenclature[dbk]{T}{Fungsi Tranpos Vektor/Matriks}

% SINGKATAN
\nomenclature[eaa]{ANN}{\textit{Artificial Neural Network}}
\nomenclature[eab]{DBT}{\textit{Dry-Bulb Temperature}}
\nomenclature[eae]{IES-VE}{\textit{Integrated Environmental Solutions - Virtual Environment}}
\nomenclature[eaf]{DTNTF}{Departemen Teknik Nuklir dan Teknik Fisika}
\nomenclature[eag]{JST}{Jaringan Saraf Tiruan}
\nomenclature[eah]{MRT}{\textit{Mean Radiant Temperature}}
\nomenclature[eai]{MAE}{\textit{Mean Absoulte Error}}
\nomenclature[eaj]{MSE}{\textit{Mean Squared Error}}
\nomenclature[eak]{NN}{\textit{Neural Network}}

