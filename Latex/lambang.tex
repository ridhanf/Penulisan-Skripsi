% PErhatikan urutan abjad kode nomenclature, misal abg, abh, cbe, cbi, menyatakan urutan penulisan di hasil akhir file .pdf. 
%Lihat contoh untuk Singkatan, meskipun urutan penulisan tidak urut abjad, misal ebh mendahului ebe, tetapi hasil akhirnya tetap urut abjad.

% LAMBANG ROMAWI
\nomenclature[aaa]{$Td$}{Suhu Ruang (Dry-Bulb Temperature)  \nomunit{$^\circ$C}}
\nomenclature[aab]{$RH$}{Kelembapan Relatif  \nomunit{\%}}
\nomenclature[aac]{$To$}{Suhu Luar (Dry-Bulb Temeperature) \nomunit{$^\circ$C}}
\nomenclature[aad]{$RD$}{Radiasi Global Matahari  \nomunit{W/m$^2$}}
\nomenclature[aae]{$AC$}{SET AC \nomunit{$^\circ$C}}
\nomenclature[aaf]{$HT$}{SET Heater \nomunit{ON}}
\nomenclature[aag]{$t$}{Waktu \nomunit{detik}}
\nomenclature[aai]{$R$}{Koefisien Korelasi \nomunit{\%}}
\nomenclature[aaj]{$\mathbb{R}$}{Domain Bilangan Riil}
\nomenclature[aar]{$x$}{Lapisan Input Jaringan Saraf Tiruan}
\nomenclature[aas]{$y$}{Lapisan Output Jaringan Saraf Tiruan}
\nomenclature[aat]{$z$}{Lapisan Tersembunyi Jaringan Saraf Tiruan}

% LAMBANG YUNANI
\nomenclature[bba]{$\nu$}{Bobot Jaringan Saraf Tiruan}
\nomenclature[bbt]{$\sigma$}{Fungsi Aktivasi Neuron}

% SUBSKRIP
%\nomenclature[cba]{i}{Iterasi Lapisan Tersembunyi}
%\nomenclature[cbb]{j}{Iterasi Input}
%\nomenclature[cbc]{l}{Iterasi Neuron}
\nomenclature[cbd]{steady-state}{Kondisi-Ajeg}

% SUPERSKRIP
\nomenclature[dbj]{n+1}{Dimensi n+1}
\nomenclature[dbk]{T}{Fungsi Tranpos Vektor/Matrix}

% SINGKATAN
\nomenclature[eaa]{ANN}{Artificial Neural Network}
\nomenclature[eab]{DBT}{Dry-Bulb Temperature}
\nomenclature[eac]{DTNTF}{Departemen Teknik Nuklir dan Teknik Fisika}
\nomenclature[ead]{IMC}{Internal Model Control}
\nomenclature[eae]{JST}{Jaringan Saraf Tiruan}
\nomenclature[eaf]{MRT}{Mean Radiant Temperature}
\nomenclature[eag]{MAE}{Mean Absoulte Error}
\nomenclature[eah]{MSE}{Mean Squared Error}
\nomenclature[eai]{NN}{Neural Network}
