\chapter{Listing Program}
\singlespacing

\lstset{language=Matlab,%
	%basicstyle=\color{red},
	breaklines=true,%
	morekeywords={matlab2tikz},
	keywordstyle=\color{blue},%
	morekeywords=[2]{1}, keywordstyle=[2]{\color{black}},
	identifierstyle=\color{black},%
	stringstyle=\color[rgb]{0.5,0,0.5},
	commentstyle=\color[rgb]{0.13,0.54,0.13},%
	showstringspaces=false,%without this there will be a symbol in the places where there is a space
	numbers=left,%
	numberstyle={\tiny \color{black}},% size of the numbers
	numbersep=9pt, % this defines how far the numbers are from the text
	emph=[1]{for,end,break},emphstyle=[1]\color{red}, %some words to emphasise
	%emph=[2]{word1,word2}, emphstyle=[2]{style},    
}

\section{Kode Sumber Program Model JST Kontroler}
\begin{lstlisting}
% Import Data
data = xlsread('DataSkripsiS1Ridhan.xlsx');
Control_Input = data(:,5:6)';
Plant_Output  = data(:,9:10)';

% Set up Data
Yp = Plant_Output;  % Plant Output
u  = Control_Input; % Manipulated Variable
[~,datasize] = size(Yp);
clear data Control_Input Load_var Plant_Output;

% Feature Scaling
parY = [30.31, 100; 16, 55.84];
[Yp, ~] = MinMaxScaler(Yp',parY);
Yp = Yp';
clear parY;

% ANN Input Output
normal = 2:datasize;
delay = 1:datasize-1;
X  = [Yp(:,delay);Yp(:,normal);u(:,delay)]; % Feature
T  = u(:,normal); % Target
clear Yp u normal delay;

% Create a Fitting Network
hiddenLayerSize = 35;
netC = feedforwardnet(hiddenLayerSize);

% Choose a Training Function
netC.trainFcn = 'trainlm'; % Levenberg-Marquardt backpropagation.

% Choose Input and Output Pre/Post-Processing Functions
% For a list of all processing functions type: help nnprocess
netC.input.processFcns = {'removeconstantrows','mapminmax'};
netC.output.processFcns = {'removeconstantrows','mapminmax'};

% Setup Division of Data for Training, Validation, Testing
% For a list of all data division functions type: help nndivision
netC.divideFcn = 'dividerand';  % Divide data randomly
netC.divideMode = 'sample';  % Divide up every sample
netC.divideParam.trainRatio = 80/100;
netC.divideParam.valRatio = 10/100;
netC.divideParam.testRatio = 10/100;

% Choose activation functions
netC.layers{1}.transferFcn = 'tansig';
netC.layers{2}.transferFcn = 'purelin';

% Choose a Performance Function
% For a list of all performance functions type: help nnperformance
netC.performFcn = 'mse';  % Mean Squared Error

% Choose Plot Functions
% For a list of all plot functions type: help nnplot
netC.plotFcns = {'plotperform','plottrainstate','ploterrhist', ...
'plotregression', 'plotfit'};

% Train the Network
[netC,tr] = train(netC,X,T);

% Test the Network
u = netC(X);

% Network Performance
e = gsubtract(T,u);
MAE = mean(abs(e),2);
MAE_All = mean(MAE);
MSE = mean(e.^2,2);
MSE_All = perform(netC,T,u);
MSE_Relatif = mean(e/T,2);
MSE_Std = std(e,0,2);

% Correlation Coefficient
[~,~,R_AC] = postreg(T(1,:),u(1,:));
[~,~,R_HT] = postreg(T(2,:),u(2,:));
[~,~,R_All] = postreg(T,u);
R = [R_AC,R_HT];
clear R_AC R_HT;

% Recalculate Training, Validation and Test Performance
trainTargets = T .* tr.trainMask{1};
valTargets = T .* tr.valMask{1};
testTargets = T .* tr.testMask{1};
All_MSETrain = perform(netC,trainTargets,u);
All_MSEVal = perform(netC,valTargets,u);
All_MSETest = perform(netC,testTargets,u);
\end{lstlisting}
\vspace{1em}

\section{Fungsi Min Max Scaler}
\begin{lstlisting}
function [newx, par] = MinMaxScaler(x,parx)
  if (parx == 0)
    newx = ( x - min(x) ) ./ ( max(x) - min(x) );
    par  = [[max(x)]; [min(x)]];
  else
    maxx  = parx(1,:);
    minx  = parx(2,:);
    newx  = ( x - minx ) ./ ( maxx - minx );
    par   = parx;
  end
end
\end{lstlisting}
\break

\section{Fungsi Kuantisasi AC}
\begin{lstlisting}
function y = QuantizationAC(u)
  AC  = round(u);
  if (AC < 12)
    y = 0;
  elseif (AC <= 16)
    y = 16;
  elseif (AC >= 30)
    y = 30;
  else
    y = AC;
end
\end{lstlisting}
\vspace{1em}

\section{Fungsi Kuantisasi Heater}
\begin{lstlisting}
function y = QuantizationHT(u)
  HT = round(u);
  if (HT < 1)
    y = 0;
  elseif (HT > 2)
    y = 2;
  else
    y = HT;
end
\end{lstlisting}
\vspace{1em}

\section{Fungsi Penskalaan Suhu Ruang}
\begin{lstlisting}
function y = ScalerTd(u)
  maxTd  = 30.31;
  minTd  = 16;
y = ( u - minTd ) ./ ( maxTd - minTd );
\end{lstlisting}
\vspace{1em}

\section{Fungsi Penskalaan Kelembapan Relatif}
\begin{lstlisting}
function y = ScalerRH(u)
  maxRH  = 100;
  minRH  = 55.84;
y = ( u - minRH ) ./ ( maxRH - minRH );
\end{lstlisting}