\chapter{Data Penelitian}

\section{Data \textit{Climate Chamber}}
Data penelitian ini dapat diakses di \textit{http://bit.ly/DataSkripsiS1Ridhan}

\section{Kinerja Model Plant JST}

\begin{table}[!h]
	\caption{Tabel Rancangan Model Plant JST Penulis}
	\label{tbl:A:NNPlantRidhan}
	\centering
	% use packages: array
	\begin{tabular}{|p{5.7cm}|p{5cm}|}
		\hline
		\textbf{Nama Hyperparameter} & \textbf{Nilai Hyperparameter} \\ \hline
		Arsitektur & Feedforward Neural Network \\ \hline
		Pembagian Data & 85\% 15\% 5\% \\ \hline 
		Jumlah Layar Tersembunyi & 1 \\ \hline
		Jumlah Neuron pada Layar & [55] \\ \hline
		Fungsi Aktivasi Layar & Hyperbolic Tangent \\ \hline
		Algoritma Pembelajaran & Levenberg-Marquardt \\ \hline
		Mean Absolute Error (MAE) & Td: 0,62$^\circ$C ; RH: 5,45\% \\ \hline
		Mean Squared Error (MSE) & Td: 0,82$^\circ$C ; RH: 54,45\% \\ \hline
		Koefisien Korelasi (R) & Td: 93,09\% ; RH: 71,44\% \\ \hline
	\end{tabular}
\end{table}
\vspace{-1em}

\section{Kinerja Emulator JST}

\begin{table}[!h]
	\caption{Tabel Rancangan Emulator JST (\textit{NN Forward Model})}
	\label{tbl:A:NNEmulator}
	\centering
	% use packages: array
	\begin{tabular}{|p{5.7cm}|p{5cm}|}
		\hline
		\textbf{Nama Hyperparameter} & \textbf{Nilai Hyperparameter} \\ \hline
		Arsitektur & Feedforward Neural Network \\ \hline
		Pembagian Data & 80\% 15\% 5\% \\ \hline 
		Jumlah Layar Tersembunyi & 1 \\ \hline
		Jumlah Neuron pada Layar & [55] \\ \hline
		Fungsi Aktivasi Layar & Hyperbolic Tangent \\ \hline
		Algoritma Pembelajaran & Levenberg-Marquardt \\ \hline
		Mean Absolute Error (MAE) & Td: 0,51$^\circ$C ; RH: 1,43\% \\ \hline
		Mean Squared Error (MSE) & Td: 0,49$^\circ$C ; RH: 5,91\% \\ \hline
		Koefisien Korelasi (R) & Td: 96,38\% ; RH: 97,79\% \\ \hline
	\end{tabular}
\end{table}

\section{Kinerja Kontroler JST}

\begin{table}[!h]
	\caption{Tabel Rancangan Kontroler JST (\textit{NN Inverse Model})}
	\label{tbl:A:NNControler}
	\centering
	% use packages: array
	\begin{tabular}{|p{5.7cm}|p{5cm}|}
		\hline
		\textbf{Nama Hyperparameter} & \textbf{Nilai Hyperparameter} \\ \hline
		Arsitektur & Feedforward Neural Network \\ \hline
		Pembagian Data & 85\% 15\% 5\% \\ \hline 
		Jumlah Layar Tersembunyi & 1 \\ \hline
		Jumlah Neuron pada Layar & [52] \\ \hline
		Fungsi Aktivasi Layar & Hyperbolic Tangent \\ \hline
		Algoritma Pembelajaran & Levenberg-Marquardt \\ \hline
		Mean Absolute Error (MAE) & AC: 0,23$^\circ$C ; HT: 0,00 \\ \hline
		Mean Squared Error (MSE) & AC: 4,85$^\circ$C ; HT: 0,00 \\ \hline
		Koefisien Korelasi (R) & AC: 98,41\% ; HT: 99,64\% \\ \hline
	\end{tabular}
\end{table}

\section{Hasil Simulasi Simulink SP1}

\begin{table}[!h]
	\caption{Hasil Simulasi Sistem Kontrol SP1}
	\label{tbl:A:SP1Ess}
	\centering
	% use packages: array
	\begin{tabular}{|l|c|c|c|}
		\hline
		\textbf{Variabel} & \textbf{SET Point} & \textbf{Output Plant} & \textbf{Error Steady-State}\\ \hline
		Suhu Ruang (Td) & 26$^\circ$C & 26,09$^\circ$C & 0,09$^\circ$C \\ \hline
		Kelembapan Relatif (RH) & 90\% & 88,76\% & 1,24\% \\ \hline
	\end{tabular}
\end{table}

\section{Hasil Simulasi Simulink SP2}

\begin{table}[!h]
	\caption{Hasil Simulasi Sistem Kontrol SP2}
	\label{tbl:A:SP2Ess}
	\centering
	% use packages: array
	\begin{tabular}{|l|c|c|c|}
		\hline
		\textbf{Variabel} & \textbf{SET Point} & \textbf{Output Plant} & \textbf{Error Steady-State}\\ \hline
		Suhu Ruang (Td) & 27$^\circ$C & 27,09$^\circ$C & 0,09$^\circ$C \\ \hline
		Kelembapan Relatif (RH) & 85\% & 86,14\% & 1,14\% \\ \hline
	\end{tabular}
\end{table}