\chapter{KESIMPULAN DAN SARAN}
\label{kesimpulan-dan-saran}

\section{Kesimpulan}
Rancangan kontroler berbasis jaringan saraf tiruan memiliki nilai \textit{steady-state error} sebesar 0,18$^\circ$C untuk suhu ruang dan sebesar 0,04\% untuk kelembapan relatif. Akan tetapi, kontroler tidak mampu mengendalikan suhu ruang \textit{climate chamber} di atas nilai \textit{set point} 26$^\circ$C dikarenakan model plant\cite{skripsiTanto} bukan merupakan model yang bergantung terhadap waktu. Kontroler berbasis jaringan saraf tiruan yang dihasilkan dibangun dengan pembagian data 80\% data latih, 10\% data validasi, dan 10\% data uji. Model Kontroler JST menggunakan fungsi aktivasi \textit{hyperbolic tangent} dengan algoritma pembelajaran Levenberg-Marquardt. Model Kontroler JST terdiri dari 1 lapisan tersembunyi dengan 35 neuron. Secara fisis, kelembapan relatif (RH) tidak dapat dikendalikan dengan sistem kontrol yang dibangun. Perubahan nilai RH yang terjadi diakibatkan oleh pengaruh AC secara tidak langsung.

\section{Saran}

%Berikut merupakan saran-saran untuk pengembangan kontroler ini agar menjadi lebih baik pada penelitian-penelitian selanjutnya:
\begin{enumerate}
	\item Memperkaya data pelatihan model JST menggunakan data pengukuran langsung pada \textit{climate chamber} dalam merancang kontroler JST.
	\item Menambahkan semacam manipulator/aktuator pada \textit{climate chamber} untuk memanipulasi kelembapan relatif ruang secara langsung seperti penelitian yang dilakukan oleh Moon \cite{paper22JJkim}. Contoh: \textit{humidifier} dan \textit{dehumidifier}.
	\item Menambahkan variabel control lainnya ke dalam rancangan kontroler, seperti kecepatan angin, suhu radian, RH lingkungan, dsb.
\end{enumerate}