\chapter{KESIMPULAN DAN SARAN}
\label{kesimpulan-dan-saran}

\section{Kesimpulan}
Rancangan kontroler berbasis jaringan saraf tiruan memiliki nilai \textit{steady-state error} sebesar 0,18$^\circ$C untuk suhu ruang. Kontroler berbasis jaringan saraf tiruan yang dihasilkan dibangun dengan pembagian data 80\% data latih, 10\% data validasi, dan 10\% data uji. Model JST Kontroler menggunakan fungsi aktivasi \textit{hyperbolic tangent}. Model JST Kontroler terdiri dari 1 lapisan tersembunyi dengan 35 neuron. Akan tetapi, kontroler tidak mampu mengendalikan suhu ruang \textit{climate chamber} di atas nilai \textit{set point} 26$^\circ$C dikarenakan data lingkungan termal dari Model IES-VE kurang mewakili kondisi sistem pada suhu yang tinggi.

\section{Saran}

%Berikut merupakan saran-saran untuk pengembangan kontroler ini agar menjadi lebih baik pada penelitian-penelitian selanjutnya:
\begin{enumerate}
	\item Disarankan untuk melakukan validasi model di nilai suhu yang tinggi terlebih dahulu apabila menggunaan model IES-VE yang serupa.
	\item Memperkaya data pelatihan model JST menggunakan data pengukuran langsung pada \textit{climate chamber} dalam merancang JST kontroler.
	\item Menambahkan semacam \textit{manipulator}/aktuator pada \textit{climate chamber} untuk memanipulasi kelembapan relatif ruang secara langsung seperti penelitian yang dilakukan oleh Moon \cite{paper22JJkim}. Contoh: \textit{humidifier} dan \textit{dehumidifier}.
	\item Menambahkan variabel kontrol lainnya ke dalam rancangan kontroler, seperti kecepatan angin, suhu radian, RH lingkungan, dsb.
\end{enumerate}