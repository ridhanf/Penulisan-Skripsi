% Ganti perintah \lipsum dengan isi intisari.
Penelitian-penelitan kenyamanan termal membutuhkan kondisi lingkungan termal pada \textit{climate chamber} (sebagai ruang uji eksperimental) untuk dapat dikondisi-kan secara otomatis sesuai dengan skema pengujian penelitian. \textit{Climate chamber} dapat terwujud jika kondisi iklim di dalamnya dapat dikendalikan sesuai dengan kebutuhan skenario penelitian. Oleh karena itu, dibutuhkan suatu sistem kontrol yang mampu mengendalikan lingkungan termal pada \textit{climate chamber}.

Penelitian ini menggunakan sampel data sebanyak 24.000 yang didapatkan dari simulasi IES-VE. Dengan menggunakan data tersebut, dibangun kontroler berbasis jaringan saraf tiruan (JST) untuk mengendalikan suhu ruang (T$_{db}$) dan kelembapan relatif (RH) pada \textit{climate chamber}. Kontroler dibangun dari model JST dengan menggunakan prinsip model invers dari model \textit{plant} berdasarkan data simulasi IES-VE. Kontroler dirancang dengan memvariasikan pembagian data pelatihan, fungsi aktivasi, serta banyak neuron pada \textit{hidden layer}. Model dipilih berdasarkan nilai \textit{mean squared errror} terkecil dari hasil variasi model. Simulasi kontrol dilakukan dengan skenario pemanasan dengan laju 0,625$^\circ$C. Kinerja hasil simulasi ditinjau melalui nilai \textit{steady-state error}.

Kontroler dibangun dengan menggunakan MATLAB dan disimulasikan deng-an menggunakan Simulink. Model JST Kontroler dibangun dengan pembagian data 80\% data latih, 10\% data validasi, dan 10\% data uji. Model JST Kontroler meng-gunakan fungsi aktivasi \textit{hyperbolic tangent} dengan algoritma pembelajaran Levenberg-Marquardt. Model JST Kontroler memiliki arsitektur jaringan dengan 1 lapisan tersembunyi (\textit{hidden layer}) berisi 35 neuron. Hasil perancangan mampu mengendalikan lingkungan termal \textit{climate chamber} dengan nilai \textit{steady-state error} sebesar 0,18$^\circ$C untuk suhu ruang dan sebesar 0,04\% untuk kelembapan relatif.
%\lipsum[100-101]

\vspace{0.4cm}
\hspace{-1.2cm}
\textbf{Kata kunci}: Lingkungan Termal, Kontroler, Jaringan Saraf Tiruan, Ruang Iklim.

