\chapter{TINJAUAN PUSTAKA}
\label{pustaka}

\section{Penelitian Lingkungan Termal \textit{Climate Chamber}}

Penelitian mengenai penggungaan \textit{climate chamber} sebagai lingkungan termal yang terkendali telah banyak dilakukan oleh peneliti-peneliti sebelumnya. Penelitian yang dilakukan meliputi berbagai bidang seperti bidang lingkungan \cite{article2.1:1} \cite{article2.1:2} \cite{article2.1:4}, bidang \textit{engineering} \cite{article2.1:5}, bidang biologi \cite{article2.1:6}\cite{article2.1:7}, dan bidang kimia \cite{article2.1:8}. Variabel lingkungan termal dalam \textit{climate chamber} berfungsi sebagai stimulan pada objek penelitian. Penelitian-penelitian tersebut dijabarkan secara lebih rinci pada Tabel \ref{tbl:2:studichamber}.

Variabel lingkungan termal yang mempengaruhi objek penelitian beragam bergantung pada tujuan dari penelitian yang akan dijalankan. Variabel yang dimaksud yaitu seperti variabel suhu \cite{article2.1:5}\cite{article2.1:6}\cite{article2.1:7}\cite{article2.1:8}\cite{article2.1:9}, kelembaban udara \cite{article2.1:8}, tekanan \cite{article2.1:4}, ataupun kombinasi dari 2 atau lebih variabel lingkungan termal \cite{article2.1:8}.

Nilai dari variabel lingkungan termal harus dapat dikendalikan sesuai dengan skenario penelitian yang akan dijalankan. Terdapat penelitian yang menginginkan nilai variabel lingkungan termal terkendali pada nilai \textit{set point} tertentu dengan akurasi yang tinggi dan distribusi yang merata pada titik-titik dalam \textit{climate chamber}. Terdapat pula penelitian yang tidak perlu memiliki pengendalian variabel lingkungan termal dengan akurasi tinggi dengan nilai eror yang masih dapat diterima, namun dengan rentang nilai yang lebar dan dapat dijaga untuk tetap berada pada rentang nilai tersebut untuk waktu yang lama. Lalu terdapat pula penelitian yang menginginkan perubahan variabel lingkungan termal dapat terjadi dengan waktu yang cepat.

Berdasarkan penelitian yang dilakukan Nur Muna Nadiya\cite{skripsiMuna}, penghuni ruang yang terbiasa terpapar kondisi lingkungan termal yang panas dan lembap mampu merasakan perubahan 1 level sensasi akibat perubahan suhu naik, minimal sebesar 2,78$^{\circ}$C dan perubahan suhu turun, minimal sebesar 2,70$^{\circ}$C. Dengan kata lain, tuntutan dari penelitian yaitu memastikan nilai variabel lingkungan suhu untuk dapat dijaga pada nilai tertentu dengan galat $\pm$ 2,7$^{\circ}$C.

\begin{landscape}
	\begin{table}[t]
		\caption{Tinjauan Pustaka Lingkungan Termal}
		\centering
		\label{tbl:2:studichamber}
		\begin{tabular}{|c|p{2.6cm}|p{3cm}|p{2.3cm}|p{5.5cm}|p{5.5cm}|}
			\hline
			Tahun & Peneliti & Lokasi Penelitian & Variabel & Fungsi Chamber & Kondisi Lingkungan Termal \\ \hline
			
			2004 \cite{article2.1:1} & H. Feriadi dan N. Hien & Bangunan tanpa sistem pendingin di Indonesia & Suhu netral, Suhu nyaman & Pengujian sensasi termal & Dilakukan pada rentang suhu 26-32,6$^{\circ}$C DBT dan 26,5-34$^{\circ}$C MRT \\ \hline
			
			2006 \cite{article2.1:2} & H. Feriadi dan N. Hien & \textit{Climate Chamber} & Sensasi termal & Pengujian sensasi termal & Metode 1: suhu 16-32$^{\circ}$C (\textit{steady state}). Metode 2: $\Delta$T = $\pm$9$^{\circ}$C (\textit{step change}) \\ \hline
			
			2007 \cite{article2.1:4} & A. P. Leskinen, J. K. Jokiniemi, dan K. E. J. Lehtinen & Partikel \textit{aging aerosol} dari pembakaran kayu & Tekanan & digunakan sebagai tempat penelitian, pengukuran, dan analisa dari proses \textit{aging flue gas} dan \textit{filtered gas} & Peneliti menginginkan tekanan di dalam chamber yang sama dengan tekanan udara di luar, dengan suhu dan kelembaban dalam chamber bukan variabel yang dikontrol namun hanya dicek berapa nilainya. \\ \hline
			
			2014 \cite{article2.1:5} & W. He, G. Xu, dan R. Shen & Pesawat ulang alik (\textit{spacecraft}) & Suhu & digunakan sebagai ruang penelitian/pengetesan yang terkontrol dari pesawat ulang alik yang mendapatkan pengaruh dari kombinasi variabel fisis suhu dan akselerasi. & Peneliti mengajukan dan menerapkan metode kontrol \textit{temperature uniformity}-nya pada \textit{chamber} penelitian dan membandingkan hasilnya dengan metode kontrol pada penelitian sebelumnya baik secara simulasi dengan Simulink maupun secara eksperimental. \\ \hline
			
			2014 \cite{article2.1:6} & A. Huguet, A. Francez, M. Dung, C. Fosse, dan S. Derenne & Lumut Sphagnum peat & Suhu & \textit{Climate chamber} / \textit{incubator} digunakan sebagai tempat penelitian dan analisa dari perubahan distribusi br GDGT pada lumut & Peneliti menginginkan suhu di dalam chamber iklim berada di 12$^{\circ}$C dan 15$^{\circ}$C. \\ \hline
		\end{tabular}
	\end{table}
\end{landscape}

\begin{landscape}
	\begin{table}[t]
		\caption{Tinjauan Pustaka Lingkungan Termal (lanjutan)}
		\centering
		\label{tbl:2:studichamber2}
		\begin{tabular}{|c|p{2.6cm}|p{3cm}|p{2.3cm}|p{5.5cm}|p{5.5cm}|}
			\hline
			
			Tahun & Peneliti & Lokasi Penelitian & Variabel & Fungsi Chamber & Perlakuan Chamber \\ \hline
			
			2016 \cite{article2.1:7} & E. Martinez, dkk. & Objek biologis, insekta/belalang & Suhu & \textit{Walk in style Temperature Controlled Chamber} (TCC) digunakan sebagai ruang penelitian dari laju proses-proses metabolisme dari insekta & Peneliti menginginkan kontrol suhu dalam chamber dengan akurasi tertentu, memiliki range atau span suhu tertentu, dan waktu yang dibutuhkan untuk mencapai set-point suhu chamber yang tidak lama. \\ \hline
			
			2018 \cite{article2.1:8} & A. Jofrereche, dkk. & Material Postcured vinyl ester resin & Suhu dan kelembapan & \textit{Weathering chamber} digunakan untuk memberikan aging pada post cured VE untuk dilihat perubahan struktur, mekanik, dan adhesive propertinya & Peneliti menginginkan terjadinya aging pada material post cured VE resin, dengan menggunakan weathering chamber yang di set pada suhu 80$^{\circ}C$, kelembaban relatif 90\% lalu dilihat pengaruh pada material tersebut pada hari ke 3, 7 dan 14 setelah dimasukan ke dalam chamber. \\ \hline
			
			2019 \cite{article2.1:9} & A. Srinivasa, dkk. & Mayat (cadaver) & Suhu & \textit{Chamber} digunakan sebagai tempat menyimpan sekaligus tempat penelitian mayat yang tersimpan dalam suhu rendah & Peneliti menjaga suhu di dalam chamber dijaga pada rentang 2$^{\circ}$C - 4$^{\circ}$C dengan pengaruh suhu panas di daerah tropis yang kecil. \\ \hline
			
			2020 \cite{skripsiMuna} & Nur Muna Nadiya & \textit{Climate Chamber} DTNTF & Suhu & \textit{Chamber} digunakan sebagai prasarana penelitian sensasi dan kenyaman termal bangunan & Suhu bervariasi dengan rentang 16-30$^{\circ}$C To (\textit{Operative Temperature}) \\ \hline
			
			2020 & Penelitian ini & \textit{Climate Chamber} DTNTF & Suhu dan kelembapan udara & \textit{Chamber} digunakan sebagai prasarana penelitian sensasi dan kenyaman termal bangunan & Suhu bervariasi dengan rentang 16-30$^{\circ}$C Td (\textit{Dry Bulb Temperature}) \\ \hline
			
		\end{tabular}
	\end{table}
\end{landscape}

\section{Sistem Kontrol Jaringan Saraf Tiruan}

Penelitian mengenai aplikasi jaringan saraf tiruan sebagai sistem kontrol telah banyak dilakukan oleh peneliti-peneliti sebelumnya. Penelitian yang dilakukan meliputi berbagai tipe bangunan seperti kantor tapak terbuka \cite{article11}, rumah/tempat tinggal \cite{article12}\cite{article13}, bangunan institusi \cite{article14}, bangunan residensial \cite{article15}, Stadium \cite{article16}, dan apartemen \cite{article17}. Variabel kontrol dalam sistem kontrol merupakan parameter yang mempengaruhi kenyamanan termal. Penelitian-penelitian tersebut dijabarkan secara lebih rinci pada Tabel \ref{tbl:2:studiANN}.

Nilai dari variabel kontrol harus dapat dikendalikan sesuai dengan skenario penelitian yang akan dijalankan. Terdapat penelitian yang menggunakan jaringan saraf tiruan secara langsung sebagai sistem kontrol. Terdapat pula penelitian yang membandingkan JST dengan metode lain, seperti logika \textit{fuzzy}, PID, RBC dan MPC. Lalu terdapat pula penelitian yang menggunakan metode lanjut dari JST, seperti NNARX, NNARMAX, NNOE \cite{article11} dan TDNN \cite{article15}. Dengan kata lain, penggunaan metode jaringan saraf tiruan untuk sistem kontrol memang sudah terbukti cukup baik.

Pada tahun 2010, G. Mustafaraj, J.Chen, dan G. Lowry melakukan penelitian yang membahas mengenai prediksi \textit{thermal behavior} dengan menggunakan Jaringan Saraf Tiruan (JST) pada kantor tapak terbuka di bangunan komersial modern. Variabel yang diukur meliputi data cuaca eksternal, suhu \textit{dry-bulb} ruang, laju kecepatan udara ventilasi, suhu udara ventilasi, dan suhu panas dan dingin air. Penelitian tersebut menggunakan 3 metode model \textit{black-box non-linear neural nerwork}, ya	itu: model \textit{neural network-based non-linear autoregressive model with external inputs} (NNARX), model \textit{ neural network-based non-linear autoregressive moving average model with external inputs} (NNMARMAX), dan model \textit{neural network-based non-linear output error} (NNOE). Semua model memberikan prediksi yang cukup baik, tetapi model NNARX dan NNARMAX mengungguli model NNOE. Nilai R$^2$ masing-masing bernilai 0.95, 0.9469, dan 0.8586 untuk NNARX, NNARMAX, dan NNOE. Penelitian tersebut menyimpulkan bahwa model NNARX lebih cocok dalam memprediksi suhu ruang menggunakan data pengembangan model dalam satu minggu selama musim \textit{summer}, \textit{autumn}, dan \textit{winter}. Model ini dapat digunakan dalam sistem kontrol HVAC dan dapat digunakan lebih luas pada jenis bangunan lainnya, termasuk rumah sakit, supermarket, bandara, dan sekolah \cite{article11}.

Pada tahun 2010, Jin Woo Moon dan Jong-Jin Kim melakukan penelitian mengenai model kontrol termal berbasis jaringan saraf tiruan untuk bangunan residensial. Tipe bangunan yang digunakan merupakan sebuah rumah di amerika. Jin Woo Moon dan Jong-Kin Kim mencoba mengendalikan kondisi termal dengan menjadikan suhu, kelembapan relatif dan PMV (\textit{Predicted Mean Vote}) sebagai variabel kontrol. Pada penelitian tersebut JST mampu memenuhi tuntutan kontrol pada variabel suhu (20-23)$^\circ$C di semua kasus, sedangkan kelembapan (35-60)\% hanya memenuhi 98\% dari total kasus yang ada \cite{article12}.

Pada tahun 2016, Jin Woo Moon, Sung Kwon Jung, Youngchul Kim, dan Seung-Hoon Han melakukan penelitian studi perbandingan metode kontrol termal bangungan berbasis jaringan saraf tiruan. Tipe bangunan yang digunakan merupakan sebuah tempat tinggal di amerika. Jin Woo Moon dan peneliti lainnya mencoba membandingkan metode kontrol ANN (JST), logika \textit{fuzzy}, dan ANFIS (\textit{adaptive neuro-fuzzy}). Pada penelitian tersebut ANN dan ANFIS lebih mendekati set point yang ditentukan (21.5$^{\circ}$C). ANN dan ANFIS memiliki nilai galat 1.13$^{\circ}$C (musim dingin) dengan nilai deviasi 1.19$^{\circ}$C untuk ANN (musim panas) dan 1.17$^{\circ}$C untuk ANFIS (musim panas) \cite{article13}.

Pada tahun 2017, Zakia Afroz, GM Shafiullah, Tania Urmee dan Gary Higgins melakukan penelitian mengenai prediksi suhu ruangan pada bangunan institusi. Penelitian tersebut menggunakan jaringan saraf tiruan untuk memprediksi suhu udara ruangan. Penelitian tersebut menegaskan bahwa mengidentifikasi variabel-variabel input yang relevan dan menyortirnya berdasarkan relevansi untuk mewakili suhu ruang dalam bangunan adalah langkah-langkah kunci untuk menentukan arsitektur jaringan yang optimal yang pada gilirannya memberikan akurasi prediksi yang baik. Untuk kedua kasus bangunan dan untuk semua set data yang berbeda yang digunakan dalam penelitian tersebut Lovenberg-Marquardt telah menemukan algoritma pelatihan yang paling cocok untuk memprediksi suhu ruang dalam ruangan dalam hal akurasi prediksi, kemampuan generalisasi dan waktu iterasi \cite{article14}.

Pada tahun 2017, Ján Drgoňa melakukan penelitian dengan membuat sebuah \textit{model predictive control} untuk rumah bertingkat 6 ruang dengan memanipulasi sistem HVAC yang ada. Dia membandingan pengendalian dengan menggunakan beberapa metode, yakni \textit{model predictive control} (MPC), PID, RBC, TDNN dan \textit{Regression Tree}. Hasil penelitian tersebut menunjukan bahwa kontroler TDNN mampu mempertahankan kenyamanan tinggi dan penghematan energi dengan kehilangan kinerja yang kecil dibandingkan MPC yg orisinil, sementara itu mampu mengurangi kompleksitas solusi secara drastis \cite{article15}.

Pada tahun 2018, Hyun-Jung Yoon, Dong-Seok Lee, Hyun Cho, dan Jae-Hun Jo melakukan penelitian mengenai prediksi lingkungan termal pada ruangan luas menggunakan jaringan saraf tiruan. Penelitian ini menjadikan stadium sebagai objek penelitiannya. Variabel yang diukur yaitu suhu permukaan tembok dalam ruang, dan suhu udara luar. Penelitian tersebut menyimpulkan bahwa metode prediksi lingkungan termal diusulkan menggunakan model JST untuk mengevaluasi lingkungan termal di ruangan besar yang dibagi menjadi zona-zona. Proses evaluasi lingkungan termal yang diturunkan dalam makalah ini dapat digunakan untuk mengontrol fasilitas HVAC di setiap zona bangunan ruang besar melalui pembelajaran mesin oleh model JST \cite{article16}.

Pada tahun 2018, Zhipeng Deng dan Qingyan Chen melakukan penelitian menggunakan jaringan saraf tiruan untuk memprediksi kenyamanan termal pada lingkungan dalam ruang dengan parameter sensasi termal dan perilaku penghuni. Bangunan yang digunakan pada penelitian tersebut berupa 10 kantor dan 10 apartemen/rumah. Variabel yang diukur meliputi suhu ruang, kelembapan relatif, insulasi pakaian, laju metabolisme tubuh, sensasi termal, dan perilaku penghuni. Model memprediksi kisaran suhu ruang dengan rentang nilai 20,6$^{\circ}$C (69$^{\circ}$F) - 25$^{\circ}$C (77$^{\circ}$F) di musim dingin dan 20,6$^{\circ}$C (69$^{\circ}$F) - 25,6$^{\circ}$C (78$^{\circ}$F) di musim panas. Perilaku penghuni mengevaluasi penerimaan lingkungan dalam ruangan dengan cara yang sama seperti sensasi termal \cite{article17}.

\begin{landscape}
	\begin{table}[hbt!]
		\caption{Tinjauan Pustaka Sistem kontrol JST}
		\label{tbl:2:studiANN}
		\centering
		\begin{tabular}{|c|p{2.8cm}|p{3cm}|p{2.8cm}|p{3.5cm}|p{7cm}|}
			\hline
			
			Tahun & Peneliti & Tipe Bangunan & Variabel kontrol & Sistem kontrol & Hasil Penelitian \\ \hline
			
			2010 \cite{article11} & G. Mustafaraj, dkk. & Kantor tapak terbuka pada bangunan komersial modern & Suhu ruang dan kelembapan relatif & Black-box no-linear neural networks: NNARX, NNARMAX, dan NNOE & Semua model memberikan prediksi yang cukup baik, tetapi model NNARX dan NNARMAX mengungguli model NNOE. Nilai R$^2$ masing-masing bernilai 0.95, 0.9469, dan 0.8586 untuk NNARX, NNARMAX, dan NNOE. \\ \hline
			
			2010 \cite{article12} & Jin Woo Moon, dkk. & Rumah, Amerika & Suhu, kelembapan relatif, dan PMV & ANN & ANN mampu memenuhi tuntutan kontrol pada variabel suhu (20-23)$^{\circ}$C di semua kasus, sedangkan kelembapan (35-60)\% hanya memenuhi 98\% dari total kasus yang ada \\ \hline
			
			2016 \cite{article13} & Jin Woo Moon, dkk. & Bangunan tempat tinggal, Amerika & Suhu dan kenyamanan termal & ANN, \textit{Fuzzy Logic}, dan ANFIS & ANN dan ANFIS lebih mendekati set point yang ditentukan (21.5$^{\circ}$C). ANN dan ANFIS memiliki nilai galat 1.13$^{\circ}$C (musim dingin) dengan nilai deviasi 1.19$^{\circ}$C untuk ANN (musim panas) dan 1.17$^{\circ}$C untuk ANFIS (musim panas). \\ \hline
			
			2017 \cite{article14} & Zakia Afroz, dkk. & Bangunan institusi & Suhu ruang & ANN & Lovenberg-Marquardt merupakan algoritma pelatihan yang paling cocok untuk memprediksi suhu ruang dalam hal akurasi prediksi, kemampuan generalisasi dan waktu iterasi untuk melatih algoritma. \\ \hline
			
		\end{tabular}
	\end{table}
\end{landscape}

\begin{landscape}
	\begin{table}[hbt!]
		\caption{Tinjauan Pustaka Sistem kontrol JST (lanjutan)}
		\label{tbl:2:studiANN2}
		\centering
		\begin{tabular}{|c|p{2.8cm}|p{3cm}|p{2.8cm}|p{3.5cm}|p{7cm}|}
			\hline
			
			Tahun & Peneliti & Tipe Bangunan & Variabel kontrol & Sistem kontrol & Hasil Penelitian \\ \hline
			
			2017 \cite{article15} & Ján Drgoňa, dkk. & Bangunan residensial 6 zona & Suhu operasional ruang & MPC, PID, RBC, dan TDNN & Kontroler TDNN mampu mempertahankan kenyamanan tinggi dan penghematan energi dengan kehilangan kinerja yang kecil dibandingkan MPC yg orisinil, sementara itu mampu mengurangi kompleksitas solusi secara drastis. \\ \hline
			
			2018 \cite{article16} & Hyun-Jung Yoon, dkk. & Zona-zona stadium & Suhu udara ruang, suhu radian rerata, dan insulasi pakaian & ANN & Proses evaluasi lingkungan termal yang diperoleh dalam penelitian ini dapat digunakan untuk mengontrol fasilitas HVAC di setiap zona bangunan ruang besar melalui pembelajaran dengan model JST. \\ \hline
			
			2018 \cite{article17} & Zhipeng Deng, dkk. & Kantor (10) dan rumah/apartemen (10) & Sensasi termal dan perilaku penghuni & ANN & Model memprediksi kisaran suhu udara dengan rentang nilai 20,6$^{\circ}$C (69$^{\circ}$F) - 25$^{\circ}$C (77$^{\circ}$F) di musim dingin dan 20,6$^{\circ}$C (69$^{\circ}$F) - 25,6$^{\circ}$C (78$^{\circ}$F) di musim panas. Perilaku penghuni mengevaluasi penerimaan lingkungan dalam ruangan dengan cara yang sama seperti sensasi termal.\\ \hline
			
			2020 & Penelitian ini & \textit{Climate Chamber} DTNTF & Suhu ruang dan kelembapan relatif & ANN & - \\ \hline
		\end{tabular}
	\end{table}
\end{landscape}
