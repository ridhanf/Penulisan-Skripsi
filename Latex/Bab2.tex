\chapter{TINJAUAN PUSTAKA}
\label{pustaka}

\section{Pengkondisian Lingkungan Termal pada \textit{Climate Chamber}}

% Yang baru
Pengkondisian lingkungan termal pada penelitian \textit{climate chamber} telah banyak dilakukan oleh peneliti-peneliti sebelumnya. Penelitian yang dilakukan meliputi bidang biologi \cite{paper21Arens} \cite{paper21JYLee} dan bidang lingkungan \cite{paper21Veronica}. Variabel lingkungan termal dalam \textit{climate chamber} berfungsi sebagai stimulan pada objek penelitian untuk meneliti sensasi dan/atau sensitivitas termal.

Pada penelitian Arens\cite{paper21Arens}, subjek yang terpapar pada lingkungan seragam disurvei untuk sensasi dan kenyamanan termal lokal dan keseluruhan (seluruh tubuh). Sensasi dan kenyamanan bagian tubuh lokal sangat bervariasi. Di lingkungan yang sejuk, tangan dan kaki terasa lebih dingin dibandingkan bagian tubuh lainnya. Kepala, tidak peka terhadap dingin tetapi peka terhadap hangat, terasa lebih hangat daripada bagian tubuh lainnya di lingkungan yang hangat. Sensasi dan kenyamanan keseluruhan mengikuti sensasi lokal (kepala) terhangat di lingkungan hangat dan terdingin (tangan dan kaki) di lingkungan sejuk. Subjek mengevaluasi kondisi netral sebagai "nyaman", tidak pernah "sangat nyaman", dan sensasi dan kenyamanan berlebihan selama perubahan langkah seluruh tubuh adalah kecil. Pada artikel ini, \textit{climate chamber} dikondisikan dengan 2 metode. Metode 1 dikonsidikan untuk berada pada suhu 16-32$^\circ$C (\textit{steady-state}). Metode 2 dikondisikan dengan perubahan step $\Delta$T = $\pm$9$^\circ$C.

Tujuan dari penelitian J. Y. Lee\cite{paper21JYLee} adalah untuk menyelidiki perbedaan etnis di ambang sensasi termal kulit dan zona sensorik antar-ambang antara tropis (Malaysia) dan penduduk asli beriklim sedang (Jepang). Hasil penelitian menunjukkan bahwa (1) laki-laki Malaysia merasakan kehangatan di dahi pada suhu kulit yang lebih tinggi (Tsk) dibandingkan laki-laki Jepang (p<0,05), sedangkan sensasi dingin pada tangan dan kaki, dirasakan pada Tsk yang lebih rendah pada orang Malaysia (p<0,05); (2) Secara keseluruhan, sensitivitas untuk mendeteksi kehangatan lebih besar di Jepang dibandingkan pria Malaysia; (3) Wilayah tubuh orang Jepang yang paling sensitif terhadap panas adalah dahi untuk pemanasan dan pendinginan, sedangkan sensitivitas termal wilayah orang Malaysia memiliki perbedaan yang lebih kecil daripada orang Jepang; (4) Perbedaan etnis di zona sensorik antar-ambang adalah terutama terlihat di dahi (1,9 $\pm$ 1,2$^\circ$C untuk orang Jepang, 3,2 $\pm$ 1,6$^\circ$C untuk orang Malaysia, p<0,05). Kesimpulannya, penduduk asli tropis cenderung merasakan hangat pada Tsk yang lebih tinggi dan lebih lambat pada kecepatan pemanasan yang sama dan memiliki jangkauan zona sensorik antar-ambang yang lebih luas daripada penduduk asli beriklim sedang. Pada artikel ini suhu \textit{climate chamber} dijaga tetap pada 28$^\circ$C \textit{operative temperature}.

Penelitian Veronica\cite{paper21Veronica} menyelidiki apakah ketika terpapar pada kondisi yang sama, orang tua (mereka yang berusia 65 ke atas) memiliki sensasi termal, kenyamanan, penerimaan, dan preferensi yang berbeda dari rekan-rekan mereka yang lebih muda. Penelitian dilakukan di ruang lingkungan kenyamanan termal, yang melibatkan 22 subjek yang lebih tua (rata-rata 69,7 tahun) dan 20 subjek yang lebih muda (29,6 tahun), terpapar pada empat kondisi pengujian antara sedikit dingin dan sedikit hangat. Persepsi kenyamanan termal subyektif untuk bagian tubuh lokal dan seluruh tubuh disurvei. Suhu kulit diukur di empat lokasi tubuh: leher, tulang belikat kanan, tangan kiri, dan tulang kering kanan. Kami juga menyelidiki korelasi antara tingkat kelemahan subjek dan tingkat kenyamanan termal mereka. Studi tersebut tidak menemukan perbedaan yang signifikan antara sensasi termal, kenyamanan, dan penerimaan subjek yang lebih tua dan yang lebih muda. Kami juga tidak menemukan korelasi antara tingkat kelemahan subjek dan sensasi termal, kenyamanan, penerimaan, dan preferensi mereka, tetapi kami tidak memiliki banyak subjek yang lemah. Pada subjek yang lebih tua dan lebih muda, suhu kulit tangan memiliki korelasi yang signifikan dengan sensasi termal lokal dan keseluruhan. Pada artikel ini suhu \textit{climate chamber} diatur pada nilai 20$^\circ$C dan 25$^\circ$C.

Berdasarkan penelitian yang dilakukan Nur Muna Nadiya\cite{skripsiMuna}, penghuni ruang yang terbiasa terpapar kondisi lingkungan termal yang panas dan lembap mampu merasakan perubahan 1 level sensasi akibat perubahan suhu naik, minimal sebesar 2,78$^{\circ}$C dan perubahan suhu turun, minimal sebesar 2,70$^{\circ}$C. Dengan kata lain, tuntutan dari penelitian yaitu memastikan nilai variabel lingkungan suhu untuk dapat dijaga pada nilai tertentu dengan galat $\pm$2,7$^{\circ}$C. 

Variabel lingkungan termal yang mempengaruhi objek penelitian beragam bergantung pada tujuan dari penelitian yang akan dijalankan. Variabel yang dimaksud yaitu seperti variabel suhu, kelembaban udara, tekanan, ataupun kombinasi dari 2 atau lebih variabel lingkungan termal. Nilai dari variabel lingkungan termal harus dapat dikendalikan sesuai dengan skenario penelitian yang akan dijalankan. Terdapat penelitian yang menginginkan nilai variabel lingkungan termal terkendali pada nilai \textit{set point} tertentu dengan akurasi yang tinggi dan distribusi yang merata pada titik-titik dalam \textit{climate chamber}. Terdapat pula penelitian yang tidak perlu memiliki pengendalian variabel lingkungan termal berakurasi tinggi dengan nilai galat yang masih dapat diterima. Akan tetapi, dituntut untuk dapat dijaga tetap berada pada rentang nilai tersebut untuk waktu yang lama. Lalu, terdapat pula penelitian yang menginginkan perubahan variabel lingkungan termal dengan waktu yang cepat. 

Pada penelitian ini, kondisi \textit{climate chamber} dituntut untuk mampu menjaga kondisi lingkungan termal pada nilai tertentu dengan galat suhu kurang dari $\pm$1$^{\circ}$C dan galat kelembapan relatif kurang dari $\pm$10\%. Penelitian-penelitian yang telah dijabarkan di atas dirangkum secara ringkas pada Tabel \ref{tbl:2:studichamber}. 

\begin{landscape}
	\begin{table}[t]
		\caption{Pengkondisian Lingkungan Termal pada \textit{Climate Chamber}}
		\centering
		\label{tbl:2:studichamber}
		\begin{tabular}{|p{1cm}|p{2.6cm}|p{3cm}|p{3cm}|p{6cm}|p{5.5cm}|}
			\hline
			Tahun & Peneliti & Lokasi Penelitian & Variabel & Fungsi Chamber & Kondisi Lingkungan Termal \\ \hline
			
			2006 \cite{paper21Arens} & E. Arens, H. Zhang, dan C. Huizenga & \textit{Climate Chamber} & Sensasi termal & \textit{Climate chamber} digunakan sebagai sarana pengujian sensasi termal & Metode 1: suhu 16-32$^{\circ}$C (\textit{steady state}). Metode 2: $\Delta$T = $\pm$9$^{\circ}$C (\textit{step change}) \\ \hline
			
			2010 \cite{paper21JYLee} & J. Y. Lee, Mohamed Saat, dkk. &\textit{Climate Chamber} & Sensitivitas termal & \textit{Climate chamber} digunakan sebagai sarana pengujian sensitivitas termal & Suhu di dalam climate chamber dijaga tetap pada 28$^{\circ}$C (\textit{Operative Temperature}) \\ \hline
			
			2019 \cite{paper21Veronica} & V. Soebarto, H. Zhang, dan S. Schiavon &\textit{Climate Chamber} & Sensasi Termal, Suhu Nyaman, Preferensi Termal & \textit{Climate chamber} digunakan sebagai sarana pengujian sensasi termal & Kondisi \textit{climate chamber} diatur pada suhu 20$^\circ$C dan 25$^\circ$C \\ \hline
			
			2020 \cite{skripsiMuna} & Nur Muna Nadiya & \textit{Climate Chamber} DTNTF FT UGM & Suhu ruang & \textit{Climate chamber} digunakan sebagai prasarana penelitian sensasi dan kenyaman termal bangunan & Suhu bervariasi naik turun (16$^{\circ}$C-30$^{\circ}$C) dengan lompatan 1$^{\circ}$C\\ \hline
			
			2020 & Penelitian ini & \textit{Climate Chamber} DTNTF FT UGM & Suhu ruang (Tdb) dan kelembapan relatif (RH) & \textit{Climate chamber} merupakan objek penelitian yang akan dikendalikan & Suhu bervariasi naik turun (16$^{\circ}$C-30$^{\circ}$C) dengan lompatan 2$^{\circ}$C\\ \hline
		\end{tabular}
	\end{table}
\end{landscape}

\section{Kontrol Jaringan Saraf Tiruan}

Penelitian mengenai aplikasi jaringan saraf tiruan sebagai kontroler telah banyak dilakukan oleh peneliti-peneliti sebelumnya. Penelitian yang dilakukan menggunakan tipe bangunan berupa rumah/tempat tinggal \cite{paper22JJkim}\cite{paper22SKJung} dan bangunan residensial \cite{paper22JanDrgona}. Variabel kontrol dalam kontroler merupakan parameter yang mempengaruhi kenyamanan termal.

Nilai dari variabel kontrol harus dapat dikendalikan sesuai dengan skenario penelitian yang akan dijalankan. Terdapat penelitian yang menggunakan jaringan saraf tiruan secara langsung sebagai kontroler. Terdapat pula penelitian yang membandingkan JST dengan metode lain, seperti logika \textit{fuzzy}, \textit{proportional–integral–derivative} (PID), ruled-based controller (RBC), \textit{model predictive control} (MPC), dan \textit{time delay neural network} (TDNN) \cite{paper22JanDrgona}. Dengan kata lain, penggunaan metode jaringan saraf tiruan untuk kontroler memang sudah terbukti cukup baik.

%Pada tahun 2010, G. Mustafaraj, J.Chen, dan G. Lowry melakukan penelitian yang membahas mengenai prediksi \textit{thermal behavior} dengan menggunakan Jaringan Saraf Tiruan (JST) pada kantor tapak terbuka di bangunan komersial modern. Variabel yang diukur meliputi data cuaca eksternal, suhu \textit{dry-bulb} ruang, laju kecepatan udara ventilasi, suhu udara ventilasi, dan suhu panas dan dingin air. Penelitian tersebut menggunakan 3 metode model \textit{black-box non-linear neural nerwork}, yaitu: model \textit{neural network-based non-linear autoregressive model with external inputs} (NNARX), model \textit{ neural network-based non-linear autoregressive moving average model with external inputs} (NNARMAX), dan model \textit{neural network-based non-linear output error} (NNOE). Semua model memberikan prediksi yang cukup baik, tetapi model NNARX dan NNARMAX mengungguli model NNOE. Nilai R$^2$ masing-masing bernilai 0.95, 0.9469, dan 0.8586 untuk NNARX, NNARMAX, dan NNOE. Penelitian tersebut menyimpulkan bahwa model NNARX lebih cocok dalam memprediksi suhu ruang menggunakan data pengembangan model dalam satu minggu selama musim panas, musim gugur, dan musim dingin. Model ini dapat digunakan dalam kontroler HVAC dan dapat digunakan lebih luas pada jenis bangunan lainnya, termasuk rumah sakit, supermarket, bandara, dan sekolah \cite{article11}.

Jin Woo Moon dan Jong-Jin Kim melakukan penelitian mengenai model kontrol termal berbasis jaringan saraf tiruan untuk bangunan residensial. Tipe bangunan yang digunakan merupakan sebuah rumah di Amerika. Jin Woo Moon dan Jong-Kin Kim mencoba mengendalikan kondisi termal dengan menjadikan suhu, kelembapan relatif dan PMV (\textit{Predicted Mean Vote}) sebagai variabel kontrol. Pada penelitian tersebut JST mampu memenuhi tuntutan kontrol pada variabel suhu (20-23)$^\circ$C di semua kasus, sedangkan kelembapan (35-60)\% hanya memenuhi 98\% dari total kasus yang ada \cite{paper22JJkim}.

Studi perbandingan metode kontrol termal bangunan berbasis jaringan saraf tiruan dilakuan oleh Jin Woo Moon, Sung Kwon Jung, Youngchul Kim, dan Seung-Hoon Han pada tahun 2016. Tipe bangunan yang digunakan merupakan sebuah tempat tinggal di Amerika. Jin Woo Moon dan peneliti lainnya mencoba membandingkan metode kontrol ANN (JST), logika \textit{fuzzy}, dan ANFIS (\textit{adaptive neuro-fuzzy}). Pada penelitian tersebut ANN dan ANFIS lebih mendekati set point yang ditentukan (21,5$^{\circ}$C untuk musim dingin dan 24,5 $^{\circ}$C untuk musim panas). ANN dan ANFIS memiliki nilai galat 0,13$^{\circ}$C (musim dingin) dengan nilai penyimpangan sebesar 0,19$^{\circ}$C untuk ANN (musim panas) dan 0,17$^{\circ}$C untuk ANFIS (musim panas) \cite{paper22SKJung}.

%Pada tahun 2017, Zakia Afroz, GM Shafiullah, Tania Urmee dan Gary Higgins melakukan penelitian mengenai prediksi suhu ruangan pada bangunan institusi. Penelitian tersebut menggunakan jaringan saraf tiruan untuk memprediksi suhu ruang. Penelitian tersebut menegaskan bahwa dengan mengidentifikasi variabel-variabel input yang relevan dan menyortirnya berdasarkan relevansi untuk mewakili suhu ruang dalam bangunan merupakan langkah-langkah kunci dalam menentukan arsitektur jaringan yang optimal yang pada gilirannya memberikan akurasi prediksi yang baik. Untuk kedua kasus bangunan dan untuk semua set data yang berbeda yang digunakan dalam penelitian tersebut, algoritma pembelajaran Levenberg-Marquardt merupakan algoritma yang paling cocok untuk memprediksi suhu ruang dalam hal akurasi prediksi, kemampuan generalisasi, dan waktu iterasi \cite{article14}.

Penelitian sistem kontrol banguanan diteliti oleh Ján Drgoňa pada rumah bertingkat dengan 6 zona ruang. Penelitian bertujuan untuk memanipulasi sistem HVAC yang ada. Sistem HVAC yang digunakan berupa radiatior yang berjumlah 1 buah di setiap ruang. Dia membandingan pengendalian dengan menggunakan beberapa metode, yakni \textit{model predictive control} (MPC), PID, RBC, TDNN dan \textit{Regression Tree}. Hasil penelitian tersebut menunjukan bahwa kontroler TDNN mampu mempertahankan kenyamanan tinggi dan penghematan energi dengan kehilangan kinerja yang kecil dibandingkan MPC yg orisinil, sementara itu TDNN mampu mengurangi kompleksitas solusi secara drastis \cite{paper22JanDrgona}.

%Pada tahun 2018, Hyun-Jung Yoon, Dong-Seok Lee, Hyun Cho, dan Jae-Hun Jo melakukan penelitian mengenai prediksi lingkungan termal pada ruangan luas menggunakan jaringan saraf tiruan. Penelitian ini menjadikan stadium sebagai objek penelitiannya. Variabel yang diukur yaitu suhu permukaan tembok dalam ruang, dan suhu lingkungan. Penelitian tersebut menyimpulkan bahwa metode prediksi lingkungan termal diusulkan menggunakan model JST untuk mengevaluasi lingkungan termal di ruangan besar yang dibagi menjadi zona-zona. Proses evaluasi lingkungan termal yang diturunkan dalam makalah ini dapat digunakan untuk mengontrol fasilitas HVAC di setiap zona bangunan ruang besar melalui pembelajaran mesin oleh model JST \cite{article16}.

%Pada tahun 2018, Zhipeng Deng dan Qingyan Chen melakukan penelitian menggunakan jaringan saraf tiruan untuk memprediksi kenyamanan termal pada lingkungan dalam ruang dengan parameter sensasi termal dan perilaku penghuni. Bangunan yang digunakan pada penelitian tersebut berupa 10 kantor dan 10 apartemen/rumah. Variabel yang diukur meliputi suhu ruang, kelembapan relatif, insulasi pakaian, laju metabolisme tubuh, sensasi termal, dan perilaku penghuni. Model memprediksi kisaran suhu ruang dengan rentang nilai 20,6$^{\circ}$C (69$^{\circ}$F) - 25$^{\circ}$C (77$^{\circ}$F) di musim dingin dan 20,6$^{\circ}$C (69$^{\circ}$F) - 25,6$^{\circ}$C (78$^{\circ}$F) di musim panas. Perilaku penghuni mengevaluasi penerimaan lingkungan dalam ruangan dengan cara yang sama seperti sensasi termal \cite{article17}.

Pada penelitian ini perancangan kontroller JST menggunakan suhu ruang dan kelembapan relatif sebagai variabel kontrol dengan menggunakan AC dan Heater sebagai pengkondisi ruang. Perancangan kontroler JST memperhitungkan variabel gangguan sistem sebagai bagian dari proses perancangan. Variabel gangguan tersebut berupa suhu lingkungan dan intensitas radiasi matahari. Penelitian-penelitian yang telah dijabarkan di atas dirangkum dengan ringkas pada Tabel \ref{tbl:2:studiANN}.

\begin{landscape}
	\begin{table}[hbt!]
		\caption{Tinjauan Pustaka Kontrol JST}
		\label{tbl:2:studiANN}
		\centering
		\begin{tabular}{|p{1cm}|p{2cm}|p{1.8cm}|p{2.7cm}|p{2.5cm}|p{3.2cm}|p{1.8cm}|p{6.7cm}|}
			\hline
			
			Tahun & Peneliti & Tipe Bangunan & Variabel kontrol & Manipulator & Variabel Gangguan & Metode Kontrol & Hasil Penelitian \\ \hline
			
			2010 \cite{paper22JJkim} & Jin Woo Moon dan Jong-Jin Kim & Rumah & Suhu, kelembapan relatif, dan PMV & AC, Heater, Humidifier, dan Dehumidifier & - & ANN & ANN mampu memenuhi tuntutan kontrol pada variabel suhu (20-23)$^{\circ}$C di semua kasus, sedangkan kelembapan (35-60)\% hanya memenuhi 98\% dari total kasus yang ada \\ \hline
			
			2011 \cite{paper22SKJung} & Jin Woo Moon, Sung Kwon Jung, dkk. & Bangunan tempat tinggal& Suhu dan kenyamanan termal & AC dan Heater & - & ANN, \textit{Fuzzy Logic}, dan ANFIS & ANN dan ANFIS lebih mendekati set point yang ditentukan. ANN dan ANFIS memiliki penyimpangan (musim dingin) sebesar 0,13$^{\circ}$C dan penyimpangan (musim panas) sebesar 0,19$^{\circ}$C untuk ANN dan 0,17$^{\circ}$C untuk ANFIS. \\ \hline
			
			2017 \cite{paper22JanDrgona} & Ján Drgoňa, dkk. & Bangunan residensial dengan 6 ruang & Suhu operasional ruang & Sistem HVAC Bangunan: 1 Radiator tiap ruang & Suhu radiasi matahari, intensitas radiasi matahari, suhu ambien, dan suhu tanah & MPC, PID, RBC, dan TDNN & Kontroler TDNN mampu mempertahankan kenyamanan tinggi dan penghematan energi dengan kehilangan kinerja yang kecil dibandingkan MPC yg orisinil, sementara itu mampu mengurangi kompleksitas solusi secara drastis. \\ \hline
			
			2020 & Penelitian ini & \textit{Climate Chamber} DTNTF FT UGM & Suhu ruang (Tdb) dan kelembapan relatif (RH) & AC dan Heater & Intensitas Radiasi Matahari dan Suhu Lingkungan & ANN & - \\ \hline
		\end{tabular}
	\end{table}
\end{landscape}