Puji dan syukur penulis panjatkan kehadirat Allah SWT yang telah memberikan rahmat dan karunia-Nya, sehingga penulis dapat menyelesaikan tugas akhir beserta penulisan skripsi ini untuk memenuhi sebagian persyaratan untuk memperoleh gelar sarjana teknik fisika.

Dalam pembuatan skripsi ini banyak kesulitan yang penulis alami terutama disebabkan oleh kurangnya pengetahuan dan sumber-sumber informasi yang terbatas. Namun berkat bimbingan dan bantuan dari semua pihak akhirnya skripsi ini dapat terselesaikan tepat pada waktunya. Oleh karena itu, penulis mengucapkan terima kasih kepada semua pihak yang telah mengarahkan dan membantu penulis dalam penyusunan skripsi ini, khususnya kepada:

\begin{enumerate}
	\item Allah SWT, atas berkat dan rahmat-Nya akhirnya penulis senantiasa diberikan kekuatan, ketabahan, dan ketenangan dalam menjalani lika-liku kehidupan.
	\item Ayah dan Ibu yang telah membesarkan, mendidik, memberikan semangat, serta doa yang tak pernah henti sehingga penulis terus bersemangat dalam menjalani kehidupan, khususnya dalam pengerjaan tugas akhir ini.
	\item Ibu Faridah selaku pembimbing utama penulis yang senantiasa memberikan masukan, arahan, dan semangat dalam mengerjakan serta menyelesaikan tugas akhir ini.
	\item Bapak Agus Arif selaku pembimbing kedua penulis yang telah memberikan masukan, arahan, dan semangat dalam mengerjakan serta menyelesaikan tugas akhir ini.
	\item Bapak Nopriadi selaku dosen pembimbing akademik penulis yang senansitasa memberikan masukan, arahan dan semangat dalam menjalani perkuliahan.
	\item Seluruh Dosen dan Staf Departemen Teknik Nuklir dan Teknik Fisika.
	\item Kerabat-kerabat dekat penulis, yakni M. Faisal Al Bantani, M. N. Fathurrahman, Salsabila K. Khansa, M. Aldan H. A., dan Irfanda Husni Sahid.	
	\item Tim TA kerabat Lab SSTK yakni Armand, Fathan, Ivan, Yerico, Shaki, Yogi, Didik, Radit, Muna, Tanto, dan Faisal.
	\item Teman-teman TF C 2015 yang senantiasa menjadi teman seperjuangan dalam menjalani kuliah selama lebih kurang 4 tahun di DTNTF FT-UGM.
	\item Serta masih banyak lagi berbagai pihak yang tidak dapat penulis sebutkan satu per satu.
\end{enumerate}

Pepatah bilang "tak ada gading yang tak retak", begitu pula dengan penulisan ini. Penulisan yang masih jauh dari kata sempurna. Oleh karena itu penulis memohon maaf apabila terdapat kekurangan ataupun kesalahan yang tertera pada skripsi ini. Kritik dan saran sangat diharapkan agar penulis dapat menulis lebih baik serta berdaya guna dimasa yang akan datang.
\newline

\begin{flushright}
Yogyakarta, \textcolor{red}{Januari 2020}
\end{flushright}
% %
\vspace{0.5cm}
\begin{flushright}
Ridhan Fadhilah
\end{flushright}