\judul{Perancangan Sistem Kontrol Lingkungan Termal \textit{Climate Chamber} Berbasis Jaringan Saraf Tiruan}

\juduleng{Design of Climate Chamber Thermal Environment Control System based on Artificial Neural Network}

\permasalahan{ Untuk memenuhi kebutuhan penelitian kenyamanan termal, kondisi lingkungan termal pada \textit{climate chamber} (sebagai ruang uji termal) haruslah dapat dikondisikan secara otomatis sesuai dengan skema pengujian penelitian.
}

\jenisTA{SKRIPSI}
\gelar{Sarjana S-1}
\prodi{Teknik Fisika}  % Teknik Nuklir atau Teknik Fisika
\prodieng{Engineering Physics}  %   Nuclear Engineering atau Engineering Physics 
\jurusan{Teknik Nuklir dan Teknik Fisika}
\nama{Ridhan Fadhilah}
\nim{15/384859/TK/43521}
\angkatan{2015}
\thselesai{2020}
\tglujian{\textcolor{red}{13 Agustus 2020}}
\tglujianeng{\textcolor{red}{August 13, 2020}}
\tglpenulisan{\textcolor{red}{10 Agustus 2020}}

\pembimbingutama{Faridah, S.T., M.Sc.}
\nippembimbingutama{19760214 200212 2 001}
\pembimbingpendamping{Ir. Agus Arif, M.T.}
\nippembimbingpendamping{196608122 199303 1 004}

\ketuasidang{Faridah, S.T., M.Sc}
\nipketuasidang{19760214 200212 2 001}
\pengujiutama{Nama Lengkap Penguji Utama}
\nippengujiutama{XXXXXXXX XXXXXX X XXX}
\anggotapenguji{Nama Lengkap Anggota Penguji}
\nipanggotapenguji{XXXXXXXX XXXXXX X XXX}

\kadep{Nopriadi, S.T., M.Sc. Ph.D}
\nipkadep{19731119 200212 1 002}



