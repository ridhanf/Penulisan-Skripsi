\judul{Perancangan Kontroler Lingkungan Termal \textit{Climate Chamber} Berbasis Jaringan Saraf Tiruan}

\juduleng{Design of ANN Based Controller for Thermal Environment of Climate Chamber}

\permasalahan{Untuk memenuhi kebutuhan penelitian kenyamanan termal, kondisi lingkungan termal pada \textit{climate chamber} (sebagai ruang uji termal) haruslah dapat dikondisikan secara otomatis sesuai dengan skema pengujian penelitian.}

\jenisTA{SKRIPSI}
\gelar{Sarjana S-1}
\prodi{Teknik Fisika}  % Teknik Nuklir atau Teknik Fisika
\prodieng{Engineering Physics}  %   Nuclear Engineering atau Engineering Physics 
\jurusan{Teknik Nuklir dan Teknik Fisika}
\nama{Ridhan Fadhilah}
\nim{15/384859/TK/43521}
\angkatan{2015}
\thselesai{2020}
\tglujian{26 Agustus 2020}
\tglujianeng{August 26th, 2020}
\tglpenulisan{28 Agustus 2020}

\pembimbingutama{Faridah, S.T., M.Sc.}
\nippembimbingutama{19760214 200212 2 001}
\pembimbingpendamping{Ir. Agus Arif, M.T.}
\nippembimbingpendamping{196608122 199303 1 004}

\ketuasidang{Faridah, S.T., M.Sc.}
\nipketuasidang{19760214 200212 2 001}
\pengujiutama{Dwi Joko Suroso, S.T., M.Eng.}
\nippengujiutama{11119880 820170 6 101}
\anggotapenguji{Sentagi Sesotya Utami, S.T., M.Sc., Ph.D.}
\nipanggotapenguji{19750226 200212 2 002}

\kadep{Nopriadi, S.T., M.Sc. Ph.D}
\nipkadep{19731119 200212 1 002}



