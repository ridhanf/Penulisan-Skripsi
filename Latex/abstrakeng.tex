% ganti perintah \lipsum dengan isi abstract
Thermal comfort studies require the thermal environment conditions in the Climate Chamber (as a thermal test room) to be automatically conditioned according to the research test scheme. By using data from the IES-VE simulation\cite{skripsiIchfan} and plant model\cite{skripsiTanto} in the previous research, this research designed a controller based on Artificial Neural Network (ANN) to control air temperature (Tdb) and relative humidity (RH) in the Climate Chamber. The Controller is designed from ANN model using the principle of plant inverse model based on IES-VE simulation data.

The ANN Controller was built using MATLAB and simulated using Simulink. ANN Controller was created by split the data into 80\% training data, 10\% validation data, and 10\% testing data. ANN controller uses the hyperbolic tangent activation function with the Levenberg-Marquardt learning algorithm. ANN Controller has a network architecture with one hidden layer containing 35 neurons. The results of the author's design able to control the thermal environment of the Climate Chamber with a steady-state error value 0.18$^\circ$C for room temperature and 0.04\% for relative humidity.
%\lipsum[148-149]

\vspace{0.5cm}
\hspace{-1.2cm}
\textbf{Keywords}: Thermal Environment, Controller, Artificial Neural Network, Climate Chamber.

