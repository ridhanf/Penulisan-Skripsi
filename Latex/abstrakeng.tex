% ganti perintah \lipsum dengan isi abstract
Thermal comfort studies require the thermal environment conditions in the climate chamber (as a thermal test room) to be automatically conditioned according to the research test scheme. Climate chamber can be realized if the climatic conditions in it can be controlled according to the needs of the research scenario. Therefore, we need a control system capable of controlling the thermal environment in the climate chamber.

This study uses a sample of 24,000 data obtained from the IES-VE simulation. By using this data, a controller based on an artificial neural network (ANN) was built to control air temperature (T$_{db}$) and relative humidity (RH) in the climate chamber. The Controller is designed from ANN model using the principle of plant inverse model based on IES-VE simulation data. Controller was designed by varying the distribution of training data, activation functions, and many neurons in the hidden layer. The model is selected based on the smallest mean squared error from the variation in the model. The control simulation is carried out with a heating scenario with a rate of 0.625$^\circ$C. The performance of the simulation results is reviewed through the steady-state error value.

The Controller was built using MATLAB and simulated using Simulink. The ANN Controller was created by spliting the data into 80\% training data, 10\% validation data, and 10\% testing data. The ANN controller uses the hyperbolic tangent activation function with the Levenberg-Marquardt learning algorithm. The ANN Controller has a network architecture with one hidden layer containing 35 neurons. The results of the design able to control the thermal environment of the climate chamber with a steady-state error value 0.18$^\circ$C for room temperature and 0.04\% for relative humidity.
%\lipsum[148-149]

\vspace{0.4cm}
\hspace{-1.2cm}
\textbf{Keywords}: Thermal Environment, Controller, Neural Network, Climate Chamber.


