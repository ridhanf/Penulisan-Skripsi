\chapter{HASIL DAN PEMBAHASAN}
\label{hasil-dan-pembahasan}
Bangunan yang dijadikan objek penelitian adalah \textit{climate chamber} DTNTF FT-UGM. Dalam bab ini, akan dibahas mengenai hasil perancangan kontroler sesuai dengan langkah-langkah yang dijelaskan pada Bab IV.

\section{Pengambilan Data Simulasi IES-VE}

\subsection{Kondisi \textit{Climate Chamber}}

Climate chamber memiliki ukuran 3m $\times$ 2m $\times$ 3m (p $\times$ l $\times$ t). Komponen-komponen di dalam climate chamber terdiri dari meja, kursi, blower, penghuni, lampu, heater, dan AC. Posisi setiap komponen di dalam \textit{climate chamber} digambarakan pada Gambar \ref{fig:5:KondisiChamber}.

\begin{figure}[!h]
	\centering
	\includegraphics[width=1\textwidth]{figures/KondisiChamber}
	\caption{Posisi Komponen-Komponen di dalam \textit{Climate Chamber}}
	\label{fig:5:KondisiChamber}
\end{figure}
%\vspace{1em}

\begin{figure}[!h]
	\centering
	\includegraphics[width=0.75\textwidth]{figures/AC}
	\caption{Perangkat AC}
	\label{fig:5:AC}
\end{figure}
%\vspace{1em}

\begin{figure}[!h]
	\centering
	\includegraphics[width=0.5\textwidth]{figures/Heater}
	\caption{Perangkat Heater}
	\label{fig:5:Heater}
\end{figure}
%\vspace{1em}

Perangkat AC yang berada di dalam \textit{Climate Chamber} DTNTF FT-UGM memiliki daya sebesar 2800W (1 PK). Perangkat AC mampu mengkondisikan lingkungan melalui aliran udara yang keluar. Maka dari itu, Perangkat AC sangatlah berpengaruh terhadap kondisi lingkungan termal di dalam ruangan. Tampak dari wujud perangkat AC dapat dilihat pada Gambar \ref{fig:5:AC}

Perangkat heater yang berada di dalam climate chamber memiliki daya sebesar 900W. Terdapat dua buah perangkat heater di dalam climate chamber. Semakin banyak perangkat heater yang aktif maka akan suhu ruang akan menjadi semakin meningkat. Kenaikan rerata suhu ruang yaitu sebesar $\pm1,9^\circ$C untuk setiap perangkat heater. Tampak dari wujud perangkat heater dapat dilihat pada Gambar \ref{fig:5:Heater}.

Selain faktor dari dalam \textit{climate chamber}, faktor dari luar ruangan \textit{climate chamber} pun secara tidak langsung mempengaruhi kondisi lingkungan termal \textit{climate chamber}. Diantaranya adalah suhu lingkungan (\textit{dry bulb temperature}) dan intensitas radiasi matahari. Posisi harian matahari mempengaruhi perubahan nilai suhu lingkungan dan intensitas radiasi matahari. Pada siang hari (posisi altitude matahari ketika berada tepat diatas \textit{climate chamber}) memberikan paparan radiasi matahari yang mengenai selubung bangunan. Hal ini menyebabkan kenaikan suhu di dalam \textit{climate chamber}. Kalor yang menembus selubung bangunan berbanding lurus dengan nilai U-value. Nilai U-Value pada selubung bangunan dapat dilihat pada Tabel \ref{tbl:5:UValue}.\\

\begin{table}[!h]
	\caption{U-Value Selubung Climate Chamber\cite{skripsiTanto}}
	\label{tbl:5:UValue}
	\centering
	% use packages: array
	\begin{tabular}{|p{5.7cm}|p{5cm}|}
		\hline
		\textbf{Selubung Climate Chamber} & \textbf{U-Value (W/m$^2$.K)} \\ \hline
		Dinding & 0,707 \\ \hline
		Atap & 1,996 \\ \hline 
		Lantai & 0,707 \\ \hline
	\end{tabular}
\end{table}
\vspace{1em}
\break

\subsection{Rancangan Skenario Pengambilan Data}
Rancangan skenario pada climate chamber menghasilkan kombinasi antara set AC dan jumlah heater ON. Set AC dikondisikan untuk menyala dari pukul 08:00 s.d. 17:00 dengan rentang nilai 16$^\circ$C - 30$^\circ$C. Set jumlah heater ON terbagi menjadi 3 kondisi, yaitu keduanya tidak menyala (berkode 0), salah satu menyala (berkode 1), dan keduanya menyala (berkode 2). Kombinasi tersebut menghasilkan 25 variasi skenario. Kombinasi set heater dan set AC digambarkan pada Gambar \ref{fig:5:HeaterAC}.

\begin{figure}[!h]
	\centering
	\includegraphics[width=0.75\textwidth]{figures/HeaterAC}
	\caption{Kombinasi SET AC dan Heater}
	\label{fig:5:HeaterAC}
\end{figure}
\vspace{1em}

Untuk variasi suhu lingkungan dan intensitas radiasi matahari digunakan 4 titik ekstrim bumi terhadap matahari yaitu pada tanggal 21 Maret, 21 Juni, 23 September dan 22 Desember. Kemudian kami melakukan simulasi disetiap titik tersebut dengan kombinasi pada Gambar \ref{fig:5:SkenarioData}. Sehingga, total skenario yang dihasilkan dari kombinasi tersebut berjumlah 100 skenario.

\begin{figure}[!h]
	\centering
	\includegraphics[width=0.85\textwidth]{figures/SkenarioData}
	\caption{Skenario Pengambilan Data}
	\label{fig:5:SkenarioData}
\end{figure}
\vspace{1em}
\break
\break

\subsection{Hasil Simulasi IES-VE}

Pada Gambar \ref{fig:5:HasilSimulasiIESVE} penulis menunjukan salah satu hasil simulasi untuk skenario SET AC 26$^\circ$C dan SET Heater ON 2 buah dengan variabel gangguan yang digambarkan pada Gambar \ref{fig:5:LoadSimulasiIESVE}. Grafik yang ditampilkan terdiri dari 4 parameter yaitu suhu luar (To), intensitas radiasi matahari (RD), suhu ruang (Tdb), dan kelembapan relatif (RH). Skenario ini dilakukan selama 24 jam dengan selang waktu pengambilan data selama 6 menit dimulai dari pukul 00:03 hingga 23:57. Selang waktu tersebut adalah waktu tersingkat yang dapat dilakukan pada software IES-VE 2019. Respon waktu suhu ruang terhadap aktivasi AC tidak penulis perhitungkan dikarenakan secara fisis, respons transien termal pada bangunan cukup lama, sehingga hanya berfokus untuk meninjau nilai kesalahan keadaan-ajeg (\textit{steady-state error}).

\begin{figure}[!h]
	\centering
	\includegraphics[width=0.9\textwidth]{figures/HasilSimulasiIESVE}
	\caption{Data Hasil Simulasi ISE-VE}
	\label{fig:5:HasilSimulasiIESVE}
\end{figure}
\vspace{1em}
\break

\begin{figure}[!h]
	\centering
	\includegraphics[width=0.9\textwidth]{figures/LoadSimulasiIESVE}
	\caption{Variabel Gangguan Simulasi ISE-VE}
	\label{fig:5:LoadSimulasiIESVE}
\end{figure}
\vspace{1em}
\break

\section{Pengembangan Model Plant JST}

Model Plant mengacu pada model JST yang telah dibangun oleh Tri Hartanto\cite{skripsiTanto} yang kemudian dikembangan kembali untuk meningkatkan kinerjanya sebagai model \textit{plant}. Arsitektur model dirancang dengan memperhatikan sistem plant pada Gambar \ref{fig:5:BlokDiagramPlant}. Arsitektur Model digambarkan pada Gambar \ref{fig:5:NNPlantModelDesign} dengan \textit{hyperparameter} yang dirangkum pada Tabel \ref{tbl:5:NNPlantTanto}. Model dikembangkan dengan melakukan variasi pembagian data (data latih, data validasi, dan data uji) ke dalam beberapa variasi. Kinerja model JST dievaluasi dengan meninjau nilai MAE dari model tersebut.

\begin{figure}[!h]
	\centering
	\includegraphics[width=0.8\textwidth]{figures/BlokDiagramPlant}
	\caption{Blok Diagram Plant}
	\label{fig:5:BlokDiagramPlant}
\end{figure}

\begin{figure}[!h]
	\centering
	\includegraphics[width=0.75\textwidth]{figures/NNPlantModelDesign}
	\caption{Arsitektur Model Plant JST}
	\label{fig:5:NNPlantModelDesign}
\end{figure}
\vspace{1em}

\begin{table}[!h]
	\caption{Tabel Rancangan Model Plant JST\cite{skripsiTanto}}
	\label{tbl:5:NNPlantTanto}
	\centering
	% use packages: array
	\begin{tabular}{|p{5cm}|p{5.2cm}|}
		\hline
		\textbf{Nama Hyperparameter} & \textbf{Nilai Hyperparameter} \\ \hline
		Arsitektur & Feedforward Neural Network \\ \hline
		Pembagian Data & 50\% 25\% 25\% \\ \hline 
		Jumlah Layar Tersembunyi & 1 \\ \hline
		Jumlah Neuron pada Layar & [55] \\ \hline
		Fungsi Aktivasi Layar & Hyperbolic Tangent \\ \hline
		Algoritma Pembelajaran & Levenberg-Marquardt \\ \hline
		Mean Absolute Error (MAE) & Tdb: 0,59$^\circ$C ; RH: 5,44\% \\ \hline
		Mean Squared Error (MSE) & Tdb: 0,75$^\circ$C ; RH: 52,33\% \\ \hline
		Koefisien Korelasi (R) & Tdb: 96,23\% ; RH: 68,90\% \\ \hline
	\end{tabular}
\end{table}

%Tri Hartanto membangun model \textit{plant} dengan menggunakan data menyeluruh selama 24 jam. Pada penelitian ini penulis membangun model dengan menggunakan data selama jam operasional, yaitu pukul 07:00 s.d. 17:00. Hal ini dilakukan karena pada praktiknya \textit{climate chamber} digunakan untuk pengujian pada jam operasional tersebut. Hasil model \textit{plant} dengan penyesuaian data ini dapat dilihat pada Tabel \ref{tbl:5:NNPlantTanto2}.

\subsection{Variasi Pembagian Data} 

Variasi pembagiaan data dilakukan dengan membandingkan beberapa variasi pembagiaan data ke dalam 5 variasi. Kemudian kinerja dari setiap pembagian data dibandingkan dengan konfigurasi \textit{hyperparameter} pada \ref{tbl:5:NNPlantTanto}.

\begin{table}[h!]
	\caption{Daftar variasi pembagian data}
	\label{tbl:5:DataSplitting}
	\centering
	% use packages: array
	\begin{tabular}{|p{3cm}|p{3cm}|p{1.5cm}|p{1cm}|p{1.5cm}|p{1cm}|}
		\hline
		Pembagian Data   & Persentase Data \\ \hline
		Data Splitting 0 & (50\% 25\% 25\%)\\ \hline
		Data Splitting 1 & (60\% 20\% 20\%)\\ \hline
		Data Splitting 2 & (70\% 15\% 15\%)\\ \hline
		Data Splitting 3 & (80\% 10\% 10\%)\\ \hline
		Data Splitting 4 & (80\% 15\% 05\%)\\ \hline
		Data Splitting 5 & (85\% 10\% 05\%)\\ \hline
	\end{tabular}
\end{table}

Pada Tabel \ref{tbl:5:DataSplitting}, "Data Splitting 0" merupakan konfigurasi pembagian data yang digunakan oleh Tri Hartanto pada penelitian sebelumnya dalam membangun model plant JST. Pada tabel yang penulis sajikan, penulis menulis pembagian data dengan format 'Data Splitting n' dan '(x\% y\% z\%)' dimana n = nomor variasi, x = pembagian data pelatihan, y = pembagian data validasi, dan z = pembagian data pengujian. Pembagian data terbaik yang penulis gunakan yaitu pembagian data bernama "Data Splitting 4". Data dibagi menjadi 3 bagian, yakni 80\% data pelatihan, 15\% data validasi, dan 5\% data pengujian.

\begin{figure}[!h]
	\centering
	\includegraphics[width=0.9\textwidth]{figures/DataSplittingResult}
	\caption{Hasil Variasi Pembagian Data}
	\label{fig:5:DataSplittingResult}
\end{figure}

\begin{figure}[!h]
	\centering
	\includegraphics[width=0.75\textwidth]{figures/DataSplittingFinal}
	\caption{Pembagian Data yang digunakan}
	\label{fig:5:DataSplittingFinal}
\end{figure}
\vspace{1em}

\subsection{Hasil Model Plant JST}

Dari pengembangan model plant JST ini, didapatkan rancangan yang lebih baik dari hasil kinerja rancangan sebelumnya. Dengan mengubah pembagiaan data dari 50\% 25\% 25\% menjadi 80\% 15\% 5\%, nilai R model untuk kelembapan relatif pun berubah dari 68,90\% menjadi sebesar 71,44\%. Hasil akhir rancangan model plant JST ini dirangkum pada Tabel \ref{tbl:5:NNPlantRidhan}.

\begin{table}[!h]
	\caption{Tabel Rancangan Model Plant JST}
	\label{tbl:5:NNPlantRidhan}
	\centering
	% use packages: array
	\begin{tabular}{|p{5cm}|p{5.2cm}|}
		\hline
		\textbf{Nama Hyperparameter} & \textbf{Nilai Hyperparameter} \\ \hline
		Arsitektur & Feedforward Neural Network \\ \hline
		Pembagian Data & 80\% 15\% 5\% \\ \hline 
		Jumlah Layar Tersembunyi & 1 \\ \hline
		Jumlah Neuron pada Layar & [55] \\ \hline
		Fungsi Aktivasi Layar & Hyperbolic Tangent \\ \hline
		Algoritma Pembelajaran & Levenberg-Marquardt \\ \hline
		Mean Absolute Error (MAE) & Tdb: 0,62$^\circ$C ; RH: 5,45\% \\ \hline
		Mean Squared Error (MSE) & Tdb: 0,82$^\circ$C ; RH: 54,45\% \\ \hline
		Koefisien Korelasi (R) & Tdb: 93,09\% ; RH: 71,44\% \\ \hline
	\end{tabular}
\end{table}

\section{Perancangan Kontroler JST}

Perancangan kontroler dipilih dengan membandingkan kinerja dengan nilai \textit{steady-state error} dari 4 rancangan kontroler berbasis jaringan saraf tiruan. Keempat rancangan kontroler tersebut adalah sebagai berikut:
\begin{enumerate}
	\item Design I: Neural Network Inverse Model
	\item Design II: NN Inverse Model dengan Umpan Variabel Manipulasi
	\item Design III: NN Inverse Model dengan Umpan Variabel Gangguan
	\item Design IV: NN Internal Model Control
\end{enumerate}

Hasil kontrol untuk suhu ruang dari empat rancangan kontroler yang digunakan dapat diamati pada Gambar \ref{fig:5:ControlComparisonTd1}, Gambar \ref{fig:5:ControlComparisonTd2}, Gambar \ref{fig:5:ControlComparisonTd3}, dan Gambar \ref{fig:5:ControlComparisonTd4}. Dari keempat rancangan tersebut dapat dilihat bahwa rancangan Design IV memiliki nilai \textit{steady-state error} paling kecil untuk variabel suhu ruang, yaitu sebesar 0,09$^\circ$C.\\

\begin{figure}[!h]
	\centering
	\includegraphics[width=0.75\textwidth]{figures/ControlComparisonTd1}
	\caption{Output Suhu Ruang Kontroler Design I}
	\label{fig:5:ControlComparisonTd1}
\end{figure}
\vspace{1em}

\begin{figure}[!h]
	\centering
	\includegraphics[width=0.75\textwidth]{figures/ControlComparisonTd2}
	\caption{Output Suhu Ruang Kontroler Design II}
	\label{fig:5:ControlComparisonTd2}
\end{figure}
\vspace{1em}

\break

\begin{figure}[!h]
	\centering
	\includegraphics[width=0.75\textwidth]{figures/ControlComparisonTd3}
	\caption{Output Suhu Ruang Kontroler Design III}
	\label{fig:5:ControlComparisonTd3}
\end{figure}
\vspace{1em}

\begin{figure}[!h]
	\centering
	\includegraphics[width=0.75\textwidth]{figures/ControlComparisonTd4}
	\caption{Output Suhu Ruang Kontroler Design IV}
	\label{fig:5:ControlComparisonTd4}
\end{figure}
\vspace{1em}

Hasil kontrol untuk kelembapan relatif dari empat rancangan kontroler yang digunakan dapat diamati pada Gambar \ref{fig:5:ControlComparisonRH1}, Gambar \ref{fig:5:ControlComparisonRH2}, Gambar, \ref{fig:5:ControlComparisonRH3}, dan Gambar \ref{fig:5:ControlComparisonRH4}. Dari keempat rancangan tersebut dapat dilihat bahwa rancangan Design IV juga memiliki nilai \textit{steady-state error} paling kecil untuk variabel kelembapan relatif, yaitu sebesar 1,24\%.\\
\break
\break

\begin{figure}[!h]
	\centering
	\includegraphics[width=0.75\textwidth]{figures/ControlComparisonRH1}
	\caption{Output Kelembapan Relatif Kontroler Design I}
	\label{fig:5:ControlComparisonRH1}
\end{figure}
\vspace{1em}

\begin{figure}[!h]
	\centering
	\includegraphics[width=0.75\textwidth]{figures/ControlComparisonRH2}
	\caption{Output Kelembapan Relatif Kontroler Design II}
	\label{fig:5:ControlComparisonRH2}
\end{figure}
\vspace{1em}

\begin{figure}[!h]
	\centering
	\includegraphics[width=0.75\textwidth]{figures/ControlComparisonRH3}
	\caption{Output Kelembapan Relatif Kontroler Design III}
	\label{fig:5:ControlComparisonRH3}
\end{figure}
\vspace{1em}

\begin{figure}[!h]
	\centering
	\includegraphics[width=0.75\textwidth]{figures/ControlComparisonRH4}
	\caption{Output Kelembapan Relatif Kontroler Design IV}
	\label{fig:5:ControlComparisonRH4}
\end{figure}
\vspace{1em}

Berdasarkan uraian diatas, dapat dilihat bahwa rancangan terbaik dengan nilai \textit{steady-state error} paling rendah adalah rancangan Design IV: NN Internal Model Control. Sehingga rancangan ini yang dipilih untuk digunakan sebagai kontroler. 

NN Internal Model Control terdiri dari 3 komponen utama, yaitu: Plant, Emulator, dan Kontroler. Emulator dibangun dari model umpan maju JST (\textit{NN Forward Model}) dan kontroler dibangun dari model umpan balik JST (\textit{NN Inverse Model}).\\

\section{Kinerja Kontroler JST Terpilih}

Kontroler JST terpilih (Design IV: NN Internal Model Control) terdiri dari 3 komponen utama, yaitu: blok plant, blok emulator, dan blok kontroler. Masing-masing dibangun oleh model jaringan saraf tiruan dengan 1 layar tersembunyi. Kinerja JST untuk Plant sudah dijelaskan pada Sub Bab 5.2. Oleh karena itu, selanjutnya akan dijelaskan kinerja model JST untuk blok emulator dan blok kontrol. Kemudian dijelaskan pula kinerja simulasi kontrol dengan 3 variasi \textit{set point}.

\subsection{Kinerja JST Blok Emulator}

Emulator JST dibangun menyerupai rancangan model plant JST. Perbedaannya berada pada masukan dan keluaran dari arsitektur JST. Emulator juga menggunakan nilai Output Plant sebelumnya sebagai masukan untuk memprediksi nilai Output Plant pada saat ini. Blok diagram untuk Emulator dapat dilihat pada Gambar \ref{fig:5:NNForwardModelDesign}. Hasil kinerja emulator JST ini dijabarkan pada Tabel \ref{tbl:5:NNEmulator}.

\begin{figure}[!h]
	\centering
	\includegraphics[width=0.5\textwidth]{figures/NNForwardModelDesign}
	\caption{Arsitektur NN Forward Model}
	\label{fig:5:NNForwardModelDesign}
\end{figure}
\vspace{-1em}

\begin{table}[!hbt]
	\caption{Tabel Rancangan Emulator JST (\textit{NN Forward Model})}
	\label{tbl:5:NNEmulator}
	\centering
	% use packages: array
	\begin{tabular}{|p{5.7cm}|p{5cm}|}
		\hline
		\textbf{Nama Hyperparameter} & \textbf{Nilai Hyperparameter} \\ \hline
		Arsitektur & Feedforward Neural Network \\ \hline
		Pembagian Data & 80\% 15\% 5\% \\ \hline 
		Jumlah Layar Tersembunyi & 1 \\ \hline
		Jumlah Neuron pada Layar & [55] \\ \hline
		Fungsi Aktivasi Layar & Hyperbolic Tangent \\ \hline
		Algoritma Pembelajaran & Levenberg-Marquardt \\ \hline
		Mean Absolute Error (MAE) & Tdb: 0,51$^\circ$C ; RH: 1,43\% \\ \hline
		Mean Squared Error (MSE) & Tdb: 0,49$^\circ$C ; RH: 5,91\% \\ \hline
		Koefisien Korelasi (R) & Tdb: 96,38\% ; RH: 97,79\% \\ \hline
	\end{tabular}
\end{table}

\subsection{Kinerja JST Blok Kontroler}

Kontroler JST dibangun dengan proses invers dari model plant JST. Pada proses pelatihan JST, dilakukan pengskalaan terhadap semua input JST menggunakan metode \textit{Min Max Scaling} kecuali variabel delay umpan masuk SET AC dan SET Heater. Pengskalaan bertujuan untuk meningkatkan kinerja JST menjadi optimal dengan menyamakan rentang nilai dan besar satuan dari setiap variabel (berupa rentang nilai dari 0 hingga 1). Masing-masing variabel diubah menjadi skala satuan dengan melakukan transformasi data secara statistik. Data dari setiap variabel akan dikurangi dengan nilai minimum variabel tersebut yang dikemudian dibagi oleh selisih dari nilai maksimum dan nilai minimum variabel tersebut. Secara lengkap dapat dirumuskan pada persamaan berikut:
\begin{equation} \label{eq:5:MinMaxScaler}
z = \frac{x_i - min(x)}{max(x) - min(x)}
\end{equation}

Rancangan kontroler JST mirip dengan rancangan model plant JST. Perbedaannya hanyalah pada jumlah neuron pada \textit{hidden layer} yang berjumlah 52 neuron. Blok diagram untuk Emulator dapat dilihat pada Gambar \ref{fig:5:NNInverseModelDesign}. Hasil kinerja dari kontroler JST ini dapat dilihat pada Tabel \ref{tbl:5:NNControler}.

\begin{figure}[!h]
	\centering
	\includegraphics[width=0.5\textwidth]{figures/NNInverseModelDesign}
	\caption{Arsitektur NN Inverse Model}
	\label{fig:5:NNInverseModelDesign}
\end{figure}

\begin{table}[!h]
	\caption{Tabel Rancangan Kontroler JST (\textit{NN Inverse Model})}
	\label{tbl:5:NNControler}
	\centering
	% use packages: array
	\begin{tabular}{|p{5.7cm}|p{5cm}|}
		\hline
		\textbf{Nama Hyperparameter} & \textbf{Nilai Hyperparameter} \\ \hline
		Arsitektur & Feedforward Neural Network \\ \hline
		Pembagian Data & 80\% 15\% 5\% \\ \hline 
		Jumlah Layar Tersembunyi & 1 \\ \hline
		Jumlah Neuron pada Layar & [52] \\ \hline
		Fungsi Aktivasi Layar & Hyperbolic Tangent \\ \hline
		Algoritma Pembelajaran & Levenberg-Marquardt \\ \hline
		Mean Absolute Error (MAE) & AC: 0,23$^\circ$C ; HT: 0,00 \\ \hline
		Mean Squared Error (MSE) & AC: 4,85$^\circ$C ; HT: 0,00 \\ \hline
		Koefisien Korelasi (R) & AC: 98,41\% ; HT: 99,64\% \\ \hline
	\end{tabular}
\end{table}

\subsection{Kinerja Simulasi Kontrol}

Kontroler terpilih diuji dengan 3 variasi kombinasi SET POINT. Kombinasi terdiri dari SET POINT untuk variabel yang diingikan dan variabel gangguan. Ketiga variasi tersebut adalah sebagai berikut:
\begin{enumerate}
	\item SP1
	\begin{table}[!h]
		\caption{Nilai Kombinasi SET POINT SP1}
		\label{tbl:5:SP1Combination}
		\centering
		% use packages: array
		\begin{tabular}{|l|l|c|}
			\hline
			\textbf{Jenis Variabel} & \textbf{Variabel} & \textbf{Nilai} \\ \hline
			SET Point & Suhu Ruang (Tdb) & 26$^\circ$C          \\ \hline
			SET Point & Kelembapan Relatif (RH) & 90\%         \\ \hline
			Variabel Gangguan & Suhu Lingkungan (To) & 27$^\circ$C           \\ \hline
			Variabel Gangguan & Radiasi Matahari (RD) & 400 W/m$^2$    \\ \hline
		\end{tabular}
	\end{table}
	
	\item SP2
	\begin{table}[!h]
		\caption{Nilai Kombinasi SET POINT SP2}
		\label{tbl:5:SP2Combination}
		\centering
		% use packages: array
		\begin{tabular}{|l|l|c|}
			\hline
			\textbf{Jenis Variabel} & \textbf{Variabel} & \textbf{SET Point} \\ \hline
			SET Point & Suhu Ruang (Tdb) & 27$^\circ$C          \\ \hline
			SET Point & Kelembapan Relatif (RH) & 85\%         \\ \hline
			Variabel Gangguan & Suhu Lingkungan (To) & 27$^\circ$C           \\ \hline
			Variabel Gangguan & Radiasi Matahari (RD) & 400 W/m$^2$    \\ \hline
		\end{tabular}
	\end{table}
	
	\item SP3
	\begin{table}[!h]
		\caption{Nilai Kombinasi SET POINT SP3}
		\label{tbl:5:SP3Combination}
		\centering
		% use packages: array
		\begin{tabular}{|l|l|c|}
			\hline
			\textbf{Jenis Variabel} & \textbf{Variabel} & \textbf{SET Point} \\ \hline
			SET Point & Suhu Ruang (Tdb) & Step 26$^\circ$C -> 27$^\circ$C \\ \hline
			SET Point & Kelembapan Relatif (RH) & Step 90\% -> 85\% \\ \hline
			Variabel Gangguan & Suhu Lingkungan (To) & 27$^\circ$C           \\ \hline
			Variabel Gangguan & Radiasi Matahari (RD) & 400 W/m$^2$    \\ \hline
		\end{tabular}
	\end{table}
\end{enumerate}

\subsubsection{SET POINT SP1}

Kombinasi SET Point dapat dilihat pada Tabel \ref{tbl:5:SP1Combination}. Hasil dari simulasi simulink dapat dilihat pada Gambar \ref{fig:5:SimulinkSP1Td} dan Gambar \ref{fig:5:SimulinkSP1RH}. Pada Gambar \ref{fig:5:SimulinkSP1Td} dan Gambar \ref{fig:5:SimulinkSP1RH} dapat dilihat bahwa nilai \textit{steady-state error} kontroler cukup kecil. Nilai \textit{steady-state error} dari simulasi ini dapat dilihat pada Tabel \ref{tbl:5:SP1Ess}. Grafik dari hasil simulasi dapat dilihat pada Gambar \ref{fig:5:SimulinkSP1Td} untuk Suhu Ruang dan Gambar \ref{fig:5:SimulinkSP1RH} untuk Kelembapan Relatif. Hasil simulasi dengan variasi SET POINT SP1 menunjukan bahwa kontroler memiliki kinerja yang cukup baik dan sudah dapat mengikuti nilai SET POINT yang diinginkan dengan nilai \textit{steady-state error} sebesar 0,09$^\circ$C untuk suhu ruang dan 1,24\% untuk kelembapan relatif. \\
%Kontroler mengeluarkan nilai \textit{Manipulated Variable} yang ditunjukkan oleh Gambar \ref{fig:5:SimulinkSP1AC} dan Gambar \ref{fig:5:SimulinkSP1HT}. 

\begin{table}[!h]
	\caption{Hasil Simulasi Kontrol SP1}
	\label{tbl:5:SP1Ess}
	\centering
	% use packages: array
	\begin{tabular}{|l|c|c|c|}
		\hline
		\textbf{Variabel} & \textbf{SET Point} & \textbf{Output Plant} & \textbf{Steady-State Error}\\ \hline
		Suhu Ruang (Tdb) & 26$^\circ$C & 26,09$^\circ$C & 0,09$^\circ$C \\ \hline
		Kelembapan Relatif (RH) & 90\% & 88,76\% & 1,24\% \\ \hline
	\end{tabular}
\end{table}

Dengan adanya emulator pada sistem kontrol, maka kontroler mampu menyesuaikan perbedaan nilai antara plant dan emulator. Rancangan ini akan menekan perbedaan antara plant dan emulator untuk mencapai nilai yang cukup kecil sehingga kontroler mampu mengendalikan suhu ruang dan kelembapan relatif sesuai dengan \textit{set point}.

Pada Gambar \ref{fig:5:SimulinkSP1Td} dan Gambar \ref{fig:5:SimulinkSP1RH} dapat dilihat bahwa kontroler mengendalikan nilai output sistem mendekati nilai \textit{set point} pada detik ke-0 hingga detik ke-25.

\begin{figure}[!h]
	\centering
	\includegraphics[width=0.75\textwidth]{figures/SimulinkSP1Td}
	\caption{Hasil Simulasi Kontrol untuk Suhu Ruang SP1}
	\label{fig:5:SimulinkSP1Td}
\end{figure}

\begin{figure}[!h]
	\centering
	\includegraphics[width=0.75\textwidth]{figures/SimulinkSP1RH}
	\caption{Hasil Simulasi Kontrol untuk Kelembapan Relatif SP1}
	\label{fig:5:SimulinkSP1RH}
\end{figure}

%\begin{figure}[!h]
%	\centering
%	\includegraphics[width=0.75\textwidth]{figures/SimulinkSP1AC}
%	\caption{Nilai MV SET AC SP1}
%	\label{fig:5:SimulinkSP1AC}
%\end{figure}

%\begin{figure}[!h]
%	\centering
%	\includegraphics[width=0.75\textwidth]{figures/SimulinkSP1HT}
%	\caption{Nilai MV SET Heater SP1}
%	\label{fig:5:SimulinkSP1HT}
%\end{figure}

\subsubsection{SET POINT SP2}

Kombinasi SET Point dapat dilihat pada Tabel \ref{tbl:5:SP2Combination}. Hasil dari simulasi simulink dapat dilihat pada Gambar \ref{fig:5:SimulinkSP2Td} dan Gambar \ref{fig:5:SimulinkSP2RH}. Pada Gambar \ref{fig:5:SimulinkSP2Td} dan Gambar \ref{fig:5:SimulinkSP2RH} dapat dilihat bahwa nilai \textit{steady-state error} kontroler cukup kecil. Nilai \textit{steady-state error} dari simulasi ini dapat dilihat pada Tabel \ref{tbl:5:SP2Ess}. Grafik dari hasil simulasi dapat dilihat pada Gambar \ref{fig:5:SimulinkSP2Td} untuk Suhu Ruang dan Gambar \ref{fig:5:SimulinkSP2RH} untuk Kelembapan Relatif. Hasil simulasi dengan variasi SET POINT SP2 menunjukan bahwa kontroler memiliki kinerja yang cukup baik dan sudah dapat mengikuti nilai SET POINT yang diinginkan dengan nilai \textit{steady-state error} sebesar 0,09$^\circ$C untuk suhu ruang dan 1,14\% untuk kelembapan relatif. \\ 
%Kontroler mengeluarkan nilai \textit{Manipulated Variable} yang ditunjukkan oleh Gambar \ref{fig:5:SimulinkSP2AC} dan Gambar \ref{fig:5:SimulinkSP2HT}.

\begin{table}[!h]
	\caption{Hasil Simulasi Kontrol SP2}
	\label{tbl:5:SP2Ess}
	\centering
	% use packages: array
	\begin{tabular}{|l|c|c|c|}
		\hline
		\textbf{Variabel} & \textbf{SET Point} & \textbf{Output Plant} & \textbf{Steady-State Error}\\ \hline
		Suhu Ruang (Tdb) & 27$^\circ$C & 27,09$^\circ$C & 0,09$^\circ$C \\ \hline
		Kelembapan Relatif (RH) & 85\% & 86,14\% & 1,14\% \\ \hline
	\end{tabular}
\end{table}

Dengan adanya emulator pada sistem kontrol, maka kontroler mampu menyesuaikan perbedaan nilai antara plant dan emulator. Rancangan ini akan menekan perbedaan antara plant dan emulator untuk mencapai nilai yang cukup kecil sehingga kontroler mampu mengendalikan suhu ruang dan kelembapan relatif sesuai dengan \textit{set point}.

Pada Gambar \ref{fig:5:SimulinkSP2Td} dan Gambar \ref{fig:5:SimulinkSP2RH} dapat dilihat bahwa kontroler mengendalikan nilai output sistem mendekati nilai \textit{set point} pada detik ke-0 hingga detik ke-125.

\begin{figure}[!h]
	\centering
	\includegraphics[width=0.75\textwidth]{figures/SimulinkSP2Td}
	\caption{Hasil Simulasi Kontrol untuk Suhu Ruang SP2}
	\label{fig:5:SimulinkSP2Td}
\end{figure}
\vspace{1em}

\begin{figure}[!h]
	\centering
	\includegraphics[width=0.75\textwidth]{figures/SimulinkSP2RH}
	\caption{Hasil Simulasi Kontrol untuk Kelembapan Relatif SP2}
	\label{fig:5:SimulinkSP2RH}
\end{figure}
\vspace{1em}
\break

%\begin{figure}[!h]
%	\centering
%	\includegraphics[width=0.75\textwidth]{figures/SimulinkSP2AC}
%	\caption{Nilai MV SET AC SP2}
%	\label{fig:5:SimulinkSP2AC}
%\end{figure}

%\begin{figure}[!h]
%	\centering
%	\includegraphics[width=0.75\textwidth]{figures/SimulinkSP2HT}
%	\caption{Nilai MV SET Heater SP2}
%	\label{fig:5:SimulinkSP2HT}
%\end{figure}

\subsubsection{SET POINT SP3}

Kombinasi SET Point dapat dilihat pada Tabel \ref{tbl:5:SP3Combination}. Hasil dari simulasi simulink dapat dilihat pada Gambar \ref{fig:5:SimulinkSP3Td} dan Gambar \ref{fig:5:SimulinkSP3RH}. Pada Gambar \ref{fig:5:SimulinkSP3Td} dan Gambar \ref{fig:5:SimulinkSP3RH} dapat dilihat bahwa nilai \textit{steady-state error} kontroler cukup kecil. Nilai \textit{steady-state error} dari simulasi ini dapat dilihat pada Tabel \ref{tbl:5:SP3Ess}. Grafik dari hasil simulasi dapat dilihat pada Gambar \ref{fig:5:SimulinkSP3Td} untuk Suhu Ruang dan Gambar \ref{fig:5:SimulinkSP3RH} untuk Kelembapan Relatif. Hasil simulasi dengan variasi SET POINT SP3 menunjukan bahwa kontroler memiliki kinerja yang cukup baik dan sudah dapat mengikuti nilai SET POINT yang diinginkan.

\begin{table}[!h]
	\caption{Hasil Simulasi Kontrol SP3}
	\label{tbl:5:SP3Ess}
	\centering
	% use packages: array
	\begin{tabular}{|l|c|c|c|}
		\hline
		\textbf{Variabel} & \textbf{SET Point} & \textbf{Output Plant} & \textbf{Steady-State Error}\\ \hline
		Suhu Ruang (Tdb) & Step 26$^\circ$C -> 27$^\circ$C & 27,09$^\circ$C & 0,09$^\circ$C \\ \hline
		Kelembapan Relatif (RH) & Step 90\% -> 85\% & 86,14\% & 1,14\% \\ \hline
	\end{tabular}
\end{table}

Dengan adanya emulator pada sistem kontrol, maka kontroler mampu menyesuaikan perbedaan nilai antara plant dan emulator. Rancangan ini akan menekan perbedaan antara plant dan emulator untuk mencapai nilai yang cukup kecil sehingga kontroler mampu mengendalikan suhu ruang dan kelembapan relatif sesuai dengan \textit{set point}.

Pada Gambar \ref{fig:5:SimulinkSP3Td} dan Gambar \ref{fig:5:SimulinkSP3RH} dapat dilihat bahwa kontroler mengendalikan nilai output sistem mendekati nilai \textit{set point} pada proses transisi fungsi step pada detik ke-250 hingga detik ke-350.

%Kontroler mengeluarkan nilai \textit{Manipulated Variable} yang ditunjukkan oleh Gambar \ref{fig:5:SimulinkSP3AC} dan Gambar \ref{fig:5:SimulinkSP3HT}.

\begin{figure}[!h]
	\centering
	\includegraphics[width=0.75\textwidth]{figures/SimulinkSP3Td}
	\caption{Hasil Simulasi Kontrol untuk Suhu Ruang SP3}
	\label{fig:5:SimulinkSP3Td}
\end{figure}
\vspace{1em}

\begin{figure}[!h]
	\centering
	\includegraphics[width=0.75\textwidth]{figures/SimulinkSP3RH}
	\caption{Hasil Simulasi Kontrol untuk Kelembapan Relatif SP3}
	\label{fig:5:SimulinkSP3RH}
\end{figure}

%\begin{figure}[!h]
%	\centering
%	\includegraphics[width=0.75\textwidth]{figures/SimulinkSP3AC}
%	\caption{Nilai MV SET AC SP3}
%	\label{fig:5:SimulinkSP3AC}
%\end{figure}

%\begin{figure}[!h]
%	\centering
%	\includegraphics[width=0.75\textwidth]{figures/SimulinkSP3HT}
%	\caption{Nilai MV SET Heater SP3}
%	\label{fig:5:SimulinkSP3HT}
%\end{figure}