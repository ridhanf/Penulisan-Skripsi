\chapter{HASIL DAN PEMBAHASAN}
\label{hasil-dan-pembahasan}
Bangunan yang dijadikan objek penelitian adalah \textit{climate chamber} DTNTF FT UGM. Dalam bab ini, akan dibahas mengenai hasil perancangan sistem kontrol sesuai dengan langkah-langkah yang dijelaskan pada Bab IV dengan memvariasikan berbagai macam masukan, kemudian mengetahui keluarannya. Variasi masukan dan keluaran akan dimodelkan dengan model jaringan saraf tiruan untuk mendapatkan parameter-parameter model yang dapat mengendalikan sistem bangunan.\\

\section{Hasil Pengembangan JST Plant}

Penulis menggunakan model JST yang telah dibangun oleh Tanto\cite{skripsiTanto} sebagai model acuan dalam penelitian ini. Model tersebut kemudian penulis kembangan kembali untuk mengingkatkan kinerjanya sebagai model \textit{plant}. \textit{Hyperparameter} yang digunakan Tanto pada pembangunan JST \textit{plant }ini dijelaskan pada tabel berikut:\\

\begin{table}[!h]
	\caption{Tabel Rancangan Model JST Plant Tanto}
	\label{tbl:5:NNPlantTanto}
	\centering
	% use packages: array
	\begin{tabular}{|p{5.7cm}|p{5cm}|}
		\hline
		\textbf{Nama Hyperparameter} & \textbf{Nilai Hyperparameter} \\ \hline
		Arsitektur & Feedforward Neural Network \\ \hline
		Pembagian Data & 50\% 25\% 25\% \\ \hline 
		Jumlah Layar Tersembunyi & 1 \\ \hline
		Jumlah Neuron pada Layar & [55] \\ \hline
		Fungsi Aktivasi Layar & Hyperbolic Tangent \\ \hline
		Algoritma Pembelajaran & Levenberg-Marquardt \\ \hline
		Mean Absolute Error & Td: 0,59$^\circ$C ; RH: 5,44\% \\ \hline
		Correlation Coefficient & Td: 96,23\% ; RH: 68,90\% \\ \hline
	\end{tabular}
\end{table}
\vspace{2em}

\noindent \textbf{Hasil Variasi Pembagian Data} 

Dalam menentukan variasi pembagiaan data, penulis melakukan perbandingan dengan beberapa variasi pembagiaan data ke dalam 5 variasi. Kemudian, penulis membandingkan kinerja dari setiap pembagian data dengan menggunakan model JST dengan konfigurasi \textit{hyperparameter} sesuai rancangan penelitian sebelumnya\cite{skripsiTanto}.
\begin{table}[h!]
	\caption{Daftar variasi pembagian data}
	\label{tbl:5:DataSplitting}
	\centering
	% use packages: array
	\begin{tabular}{|p{3cm}|p{3cm}|p{1.5cm}|p{1cm}|p{1.5cm}|p{1cm}|}
		\hline
		Pembagian Data   & Persentase Data \\ \hline
		Data Splitting 0 & (50\% 25\% 25\%)\\ \hline
		Data Splitting 1 & (60\% 20\% 20\%)\\ \hline
		Data Splitting 2 & (70\% 15\% 15\%)\\ \hline
		Data Splitting 3 & (80\% 10\% 10\%)\\ \hline
		Data Splitting 4 & (80\% 15\% 05\%)\\ \hline
		Data Splitting 5 & (85\% 10\% 05\%)\\ \hline
	\end{tabular}
\end{table}

\begin{figure}[!h]
	\centering
	\includegraphics[width=0.9\textwidth]{figures/DataSplittingResult}
	\caption{Hasil Variasi Pembagian Data}
	\label{fig:5:DataSplittingResult}
\end{figure}

\begin{figure}[!h]
	\centering
	\includegraphics[width=1\textwidth]{figures/DataSplittingFinal}
	\caption{Pembagian Data yang digunakan}
	\label{fig:5:DataSplittingFinal}
\end{figure}

Pada Tabel \ref{tbl:5:DataSplitting}, "Data Splitting 0" merupakan konfigurasi pembagian data yang digunakan oleh Tanto pada penelitian sebelumnya dalam membangun model JST \textit{plant}. Pada tabel yang penulis sajikan, penulis menulis pembagian data dengan format 'Data Splitting n' dan '(x\% y\% z\%)' dimana n = nomor variasi, x = pembagian data pelatihan, y = pembagian data validasi, dan z = pembagian data pengujian. Pembagian data terbaik yang penulis gunakan yaitu pembagian data bernama "Data Splitting 4". Data dibagi menjadi 3 bagian, yakni 80\% data pelatihan, 15\% data validasi, dan 5\% data pengujian. Sehingga didapatkan rancangan terbaik penulis yang dirangkum pada Tabel \ref{tbl:5:NNPlantRidhan} di bawah ini.

\begin{table}[!h]
	\caption{Tabel Rancangan Model JST Plant Penulis}
	\label{tbl:5:NNPlantRidhan}
	\centering
	% use packages: array
	\begin{tabular}{|p{5.7cm}|p{5cm}|}
		\hline
		\textbf{Nama Hyperparameter} & \textbf{Nilai Hyperparameter} \\ \hline
		Arsitektur & Feedforward Neural Network \\ \hline
		Pembagian Data & 80\% 15\% 05\% \\ \hline 
		Jumlah Layar Tersembunyi & 1 \\ \hline
		Jumlah Neuron pada Layar & [55] \\ \hline
		Fungsi Aktivasi Layar & Hyperbolic Tangent \\ \hline
		Algoritma Pembelajaran & Levenberg-Marquardt \\ \hline
		Mean Absolute Error & Td: 0,62$^\circ$C ; RH: 5,45\% \\ \hline
		Correlation Coefficient & Td: 93,09\% ; RH: 71,44\% \\ \hline
	\end{tabular}
\end{table}

\section{Hasil Perancangan Sistem Kontrol JST}

Pada sub-bab ini penulis akan menjabarkan hasil perancangan sistem kontrol dimulai dari kinerja JST Blok Kontroler (NN Inverse Model), Kinerja JST Internal Model (NN Forward Model), dan kinerja dari sistem kontrol pada simulasi simulink.

\subsection{Kinerja JST Blok Kontroler}

Pada proses pelatihan JST, penulis melakukan pengskalaan terhadap semua input JST menggunakan metode \textit{Min Max Scaling} kecuali variabel delay umpan masuk SET AC dan SET Heater. Pengskalaan bertujuan untuk meningkatkan kinerja JST menjadi optimal dengan menyamakan rentang dan besar satuan dari setiap variabel. Masing-masing variabel diubah menjadi skala satuan dengan melakukan transformasi data secara statistik. Data dari setiap variabel akan dikurangi dengan nilai minimum variabel tersebut yang dikemudian dibagi oleh selisih dari nilai maksimum dan nilai minimum variabel tersebut. Secara lengkap dapat dirumuskan pada persamaan berikut:
\begin{equation} \label{eq:5:MinMaxScaler}
z = \frac{x_i - min(x)}{max(x) - min(x)}
\end{equation}

\begin{figure}[!h]
	\centering
	\includegraphics[width=0.5\textwidth]{figures/NNInverseModelDesign}
	\caption{Arsitektur NN Inverse Model}
	\label{fig:5:NNInverseModelDesign}
\end{figure}

\begin{table}[!h]
	\caption{Tabel Rancangan JST Blok Kontroler Tanto}
	\label{tbl:5:NNcontrol}
	\centering
	% use packages: array
	\begin{tabular}{|p{5.7cm}|p{5cm}|}
		\hline
		\textbf{Nama Hyperparameter} & \textbf{Nilai Hyperparameter} \\ \hline
		Arsitektur & Feedforward Neural Network \\ \hline
		Pembagian Data & 80\% 15\% 5\% \\ \hline 
		Jumlah Layar Tersembunyi & 1 \\ \hline
		Jumlah Neuron pada Layar & [52] \\ \hline
		Fungsi Aktivasi Layar & Hyperbolic Tangent \\ \hline
		Algoritma Pembelajaran & Levenberg-Marquardt \\ \hline
		Mean Absolute Error & AC: 0,21$^\circ$C ; HT: 0,00\% \\ \hline
		Correlation Coefficient & AC: 98,37\% ; HT: 99,58\% \\ \hline
	\end{tabular}
\end{table}

\subsection{Kinerja JST Internal Model}

\begin{figure}[!hbt]
	\centering
	\includegraphics[width=0.5\textwidth]{figures/NNForwardModelDesign}
	\caption{Arsitektur NN Forward Model}
	\label{fig:5:NNForwardModelDesign}
\end{figure}

\begin{table}[!hbt]
	\caption{Tabel Rancangan JST Blok Kontroler Tanto}
	\label{tbl:5:NNInternalModel}
	\centering
	% use packages: array
	\begin{tabular}{|p{5.7cm}|p{5cm}|}
		\hline
		\textbf{Nama Hyperparameter} & \textbf{Nilai Hyperparameter} \\ \hline
		Arsitektur & Feedforward Neural Network \\ \hline
		Pembagian Data & 80\% 15\% 5\% \\ \hline 
		Jumlah Layar Tersembunyi & 1 \\ \hline
		Jumlah Neuron pada Layar & [52] \\ \hline
		Fungsi Aktivasi Layar & Hyperbolic Tangent \\ \hline
		Algoritma Pembelajaran & Levenberg-Marquardt \\ \hline
		Mean Absolute Error & Td: 0.51$^\circ$C ; RH: 1.43\% \\ \hline
		Correlation Coefficient & Td: 96,38\% ; RH: 97,79\% \\ \hline
	\end{tabular}
\end{table}

\subsection{Kinerja Diagram Blok Sistem Kontrol}

\begin{figure}[!h]
	\centering
	\includegraphics[width=1\textwidth]{figures/ControlDesignDiagram}
	\caption{Blok Diagram Sistem Kendali JST}
	\label{fig:5:ConstrolSystemBlockDiagram}
\end{figure}
