\chapter{HASIL DAN PEMBAHASAN}
\label{hasil-dan-pembahasan}

\section{Hasil Pengambilan Data}
Hasil-hasil yang disajikan bukan data mentah, melainkan data yang telah diolah dengan proses sebagaimana tercantum dalam pasal ``Rencana analisis hasil'' Bab IV tentang ``Pelaksanaan Penelitian''.
\subsection{Kondisi \textit{Climate Chamber}}
\subsection{Hasil Rancangan Skenario}
\subsection{Hasil Simulasi IES-VE}

\section{Pembangunan Arsitektur JST}
Data dibagi menjadi 3 bagian, yakni 70\% data pelatihan, 15\% data validasi, dan 15\% data pengujian. Model JST menggunakan arsitektur \textit{multilayer perceptron} dengan jumlah neuron sebanyak $x_1$ di lapisan tersembunyi 1, $x_2$ di lapisan tersembunyi 2, dan $x_3$ di lapisan tersembunyi 3.

\section{Analisis Kinerja Arsitektur JST yang terpilih}

Persamaan ditulis rata tengah dan nomor persamaan ditulis rata kanan. Nomor persamaan
diurutkan dengan format (nomor\_bab.nomor\_persamaan). Contoh dapat dilihat pada Persamaan \eqref{eq:1}.

\begin{equation}
    \dfrac{Dv}{Dt} = \dfrac{\partial v}{\partial t} + \nabla \cdot \mathbf{uu}
\label{eq:1}
\end{equation}

